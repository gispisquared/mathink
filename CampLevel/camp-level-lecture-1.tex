\documentclass{article}
\usepackage[inline]{asymptote}
\usepackage{enumitem,amsmath,amsfonts,geometry,parskip,amssymb}
\def\asydir{asy}
\title{Induction and Divisibility}
\author{Andres Buritica}
\date{April 16, 2022}
\begin{document}
\maketitle
\section{Induction and variants}
  Arguably, the defining property of the integers is the \textbf{Principle of
    Mathematical Induction}: if we have a set $S\subseteq\mathbb N$ such that
    $\forall a\in S,\ a+1\in S$ then $S=\mathbb N$.
  
  To prove that some sentence $P(n)$ is true for all positive integers $n$, we
  follow the following structure.
  \begin{itemize}
    \item Prove that $P(1)$ is true.
    \item Prove that if $P(n)$ is true, then $P(n+1)$ is true.
  \end{itemize}
  If we've done both of those things, why can we conclude that $P(a)$ is true
  for all $a\in\mathbb N$?

  We may use induction to prove some foundational results about the integers:
  \begin{itemize}
    \item For all positive integers $n$, $n\ge 1$.
    \item If $S$ is a nonempty set of positive integers, there is some $a\in
      S$ such that for any $b\in S$ we have $a\le b$.
      This is known as the \textbf{Well-Ordering Principle}.
  \end{itemize}

  The final addition to our induction toolkit will be \textbf{strong induction}:
  if $S$ is a set of positive integers such that $1\in S$ and
      \[\forall a\in \mathbb N,\ (\forall b\in\mathbb N,\ b\le a\implies b\in
        S)\implies a+1\in S,\]
        then $\mathbb N=S$.
\section{Divisibility}
  For integers $a$ and $b$, we say $a\mid b$ (read ``$a$ divides
  $b$'') if there is some integer $c$ with $b=a\times c$.
  \begin{itemize}
    \item Prove that if $a\mid b$ and $a\mid c$, then for any integers $m$ and
      $n$, $a\mid bm+cn$.
    \item Prove that if $a$ and $b$ are positive integers with $a\mid b$ then
      $a\le b$.
  \end{itemize}

  We define a \emph{prime} as a positive integer larger than 1 which is not
  divisible by any positive integer other than 1 and itself.
  \begin{itemize}
    \item Prove that every positive integer larger than 1 has a prime factor.
    \item Prove that there are infinitely many primes.
  \end{itemize}
\section{GCD, Euclid, Bezout}
  We define the \emph{greatest common divisor} of two integers $a$ and $b$, not
  both of which are 0, as the largest positive integer $d$ such that $d\mid a$
  and $d\mid b$. We notate it by $\gcd(a,b)$. For convenience we also define
  $\gcd(0,0)=0$.

  Similarly, the \emph{least common multiple} $\mathrm{lcm}(a,b)$ is the
  smallest positive integer $l$ such that $a\mid l$ and $b\mid l$. For
  convenience we also define $\gcd(0,a)=0$.
  \begin{itemize}
    \item Let $a$ be an integer and let $b$ be a positive integer. Prove that
      there is exactly one pair of integers $(q,r)$ with $0\le r<b$ such that
      $a=qb+r$.
    \item Prove that for any integers $a,b,q,r$, if $a=qb+r$ then
      $\gcd(a,b)=\gcd(b,r)$.
    \item Let $a$ and $b$ be integers. Prove that there are integers $c$ and $d$
      such that $ac+bd=\gcd(a,b)$.
    \item Let $a$ and $b$ be integers with $\gcd(a,b)=1$, and let $c$ be an
      integer. Prove that if $a\mid c$ and $b\mid c$ then $ab\mid c$.
    \item Assume that $p_1,\ldots,p_m,\ q_1,\ldots,q_n$ are primes such that
      \[p_1p_2\cdots p_m=q_1q_2\cdots q_n.\]
      Prove that the $q_i$s are a permutation of the $p_i$s.
  \end{itemize}
\section{Prime Factorisations}
  Therefore, each positive integer has a unique prime factorisation (the
  Fundamental Theorem of Arithmetic).
  In particular we can write a positive integer $n$ uniquely as
  \[n=p_1^{e_1}p_2^{e_2}\cdots p_k^{e_k},\]
  where $p_i$ are all prime and $e_i$ are all positive integers.

  Prime factorisations allow us to view statements about divisibility and
  multiplication in terms of the exponents $e_i$.

  In what follows, let 
  \[a=p_1^{e_1}p_2^{e_2}\cdots p_m^{e_m},\
        b=q_1^{f_1}q_2^{f_2}\cdots q_k^{e_k}.\]
  \begin{itemize}
    \item Prove that $a\mid b$ if and only if for each $i$ we have that $p_i=q_j$
      for some $j$, and that $e_i\le f_j$.
    \item Prove that $a$ is a perfect $k$th power if and only if $k\mid e_i$ for all $i$.
    \item Prove that the lcm is found by taking the maximum power of each prime that
      divides either $a$ or $b$, while the gcd is found by taking the minimum power of
      each prime that divides both $a$ and $b$.
    \item Prove that $\gcd(a,b)\times\mathrm{lcm}(a,b)=ab$.
  \end{itemize}
\section{Strategies}
  \begin{itemize}
    \item Take out the gcd: that is, if you have two integers $a$ and $b$ you
      can write $a=dx,\ b=dy$ where $x$ and $y$ are coprime.
    \item Factorisations: two especially useful ones are
      \begin{align*}
        axy+bx+cy=d\iff (ax+c)(ay+b)=ad+bc, \\
        a^k-b^k=(a-b)\left(a^{k-1}+a^{k-2}b+\cdots+b^{k-1}\right).
      \end{align*}
  \end{itemize}
\section{Problems}
  \begin{enumerate}
    \item Given positive integers $n>1$ and $k$, prove that there are unique
      nonnegative integers $m,a_0,a_1,\ldots,a_m$ such that $a_m>0,\ 0\le
      a_i<n$ for all $i$, and
      \[k=\sum_{i=0}^m a_i n^i.\]
    \item Find all integers $n$ such that $n^2+1$ divides $n^3+n^2-n-15$.
    \item Find all right-angled triangles with positive integer sides 
      such that their area and perimeter are equal.
    \item Let $a,m,n$ be positive integers with $a>1$.
      Prove that $\gcd(a^m-1,a^n-1)=a^{\gcd(m,n)}-1$.
    \item Prove that $1^k+2^k+\cdots+n^k$ is divisible by $1+2+\cdots+n$ for
      all positive integers $n$ and odd positive integers $k$.
    \item Let $a,b,c,d$ be positive integers with $ab=cd$. Prove that there
      exist positive integers $p,q,r,s$ such that \(a=pq,b=rs,a=pr,d=qs\).
    \item Let $p$ be a prime with $p>3$. Prove that there are positive integers
      $a<b<\sqrt p$ such that $p-b^2\mid p-a^2$.
    \item Find all pairs of positive integers $a,b$ such that \[b^2-a\mid
        a^2+b\quad\text{and}\quad a^2-b\mid b^2+a.\]
    \item Find all pairs of positive integers $x,y$ such that $xy^2+y+7\mid
      x^2y+x+y$.
    \item Prove that for any nonnegative integer $n$, the number $7^{7^n}+1$ is
      the product of at least $2n+3$ (not necessarily distinct) primes.
  \end{enumerate}
\newpage
\section{Homework}
  \begin{enumerate}
    \item Prove that every positive integer can be uniquely represented as a sum
      of one or more Fibonacci numbers such that the sum does not include two
      consecutive Fibonacci numbers.
    \item Let $a, b, c$ be positive integers with $a^2+b^2=c^2$, such that no
      positive integer larger than 1 divides all of them. Prove that
      there exist positive integers $x,y,z$ such that $a,b,c$ equal
      $x^2-y^2,2xy,x^2+y^2$ in some order.
    \item
      \begin{enumerate}
        \item Find all integers $a,b,c$ with $1<a<b<c$ such that $(a-1)(b-1)(c-1)$
          divides $abc-1$.
        \item Do there exist distinct prime numbers $a,b,c$ such that
          \[a\mid bc+b+c,\ b\mid ac+a+c,\ c\mid ab+a+b?\]
      \end{enumerate}
  \end{enumerate}
\end{document}
