\documentclass{article}
\usepackage[inline]{asymptote}
\usepackage{enumitem,amsmath,amsfonts,amssymb,geometry,parskip}
\def\asydir{asy}
\title{Constructions and Existence Proofs}
\author{Andres Buritica Monroy}
\date{}
\begin{document}
\maketitle
\section{Existence Results}
Number theory has quite a few results about the existence of a number satisfying
certain properties that you should be aware of. We've already seen many of
these:
\begin{itemize}
  \item There are infinitely many primes.
  \item (Bezout's Identity)
    For any coprime integers $a$ and $b$, there are integers $x$ and $y$ such
    that $ax+by=1$.
  \item (Chinese Remainder Theorem)
    For any positive integers $a_1,\ldots,a_n$, and any pairwise coprime positive
    integers $b_1,\ldots,b_n$, there is exactly one residue $x\pmod {b_1\ldots
    b_n}$ such that $x\equiv a_i\pmod{b_i}$ for each $i$.
  \item There exists a generator mod $p^k,2p^k$ for any prime $p$ and positive
    integer $k$.
  \item (Pell Equations) For any positive integer $d$, there are infinitely many
    pairs $(x,y)$ of positive integers such that $x^2-dy^2=1$.
  \item (Hensel's Lemma) If $P$ is a polynomial with integer coefficients, and
    $r$ and $n$ are positive integers such that $n\mid P(r)$ but
    $\gcd(n,P'(r))=1$, then for any positive integer $k$ there is a positive
    integer $r_k$ such that $n^k\mid P(r_k)$.
\end{itemize}
The pigeonhole principle isn't strictly a number-theoretic result but is also
often useful (see e.g.\ density arguments).
Here are a few other results that may be helpful:
\begin{itemize}
  \item (Dirichlet's Theorem)
    For any coprime positive integers $a$ and $b$, there are infinitely many
    positive integers $k$ such that $a+bk$ is prime.
  \item (Bertrand's Postulate)
    For any positive integer $n$, there is a prime in $(n,2n]$. %chktex 9 
  \item (Fermat's Christmas Theorem) Let $p$ be prime. There are positive
    integers $a$ and $b$ such that $p=a^2+b^2$ unless $p\equiv 3\pmod 4$.
  \item (Schur's Theorem) For any polynomial $p$ with integer coefficients, the
    set of primes that divide $p(x)$ for some $x$ is infinite.
  \item (Zsigmondy's Theorem) If $a>b>0$ are coprime integers, then for any
    integer $n\ge 3$ there is a prime number $p$ that divides $a^n-b^n$ and does
    not divide $a^k-b^k$ for any positive integer $k<n$, unless
    $(a,b,n)=(2,1,6)$. The same holds for $a^n+b^n$ with the exception
    $2^3+1^3=9$.
\end{itemize}
\section{Advice}
You know the drill --- get your hands dirty and try small cases. For many of
these problems, there is some property of the construction you want to control.
Remember properties like Fermat/Euler and Wilson that allow you to control stuff.
CRT is especially useful because it allows
you to combine a bunch of modular conditions into one. Most of the time
Dirichlet then gives you a prime for free.

Sometimes you just have to try a bunch of stuff until something magically works.
\section{Problems}
\begin{enumerate}
  \item Prove that if $n$ is not a multiple of $4$, then there are positive
    integers $a$ and $b$ such that $n\mid a^2+b^2+1$.
  \item Let $s(n)$ be the sum of the digits of $n$. Prove that for each positive
    integer $k$ there exists a positive integer $n$ such that $n+s(n)$ equals
    either $k$ or $k+1$.
  \item Let $a$ and $b$ be positive integers. Prove that there are infinitely
    many positive integers $n$ such that $n^b-1\nmid a^n+1$.
  \item Let $a,b,c$ be pairwise coprime positive integers. Prove that there
    exist infinitely many triples $x,y,z$ of distinct positive integers such
    that $x^a+y^b=z^c$.
  \item Prove that there exists a positive integer divisible by $2^{2023}$ whose
    decimal representation does not contain any zeros.
  \item Prove that there are infinitely many positive integers $n$ such that
    $d(n)$ and $\varphi(n)$ are both squares.
  \item Prove that there are infinitely many pairs $a$ and $b$ of perfect
    squares such that they have the same number of digits in decimal, and their
    concatenation is also a square.
  \item Prove that there exist infinitely many positive integers $n$ such that
    $n^2+1\mid n!$.
  \item For which positive integers $r$ and $s$ does there exist a positive
    integer $n$ such that $nr$ and $ns$ have the same number of divisors?
\end{enumerate}
\newpage
\section{Homework}
\begin{enumerate}
  \item Prove that there are infinitely many distinct pairs $a,b$ of coprime
    integers such that $a>1,b>1$ and $a+b\mid a^b+b^a$.
  \item Prove that for each positive integer $k$ there exists an arithmetic
    sequence of $k$ positive rational numbers such that when they are written in lowest
    terms, all numerators and denominators are pairwise distinct.
  \item Prove that there exists a positive integer $m$ such that the equation
    $\varphi(n)=m$ has at least $2023$ solutions $n$.
\end{enumerate}
\end{document} %chktex 17 
