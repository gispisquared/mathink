\documentclass{article}
\usepackage[inline]{asymptote}
\usepackage{enumitem,amsmath,amsfonts,amssymb,geometry,parskip}
\def\asydir{asy}
\title{Integer Polynomials}
\author{Andres Buritica}
\date{}
\begin{document}
\maketitle
\section{Basic facts}
For all of these, let $p(x)=\sum_{i=0}^n a_i x^i$ be a polynomial with integer
coefficients. 

We say that $p(x)$ is \emph{divisible} by a polynomial $q(x)$ if there is a
polynomial $r(x)$ (not necessarily with integer coefficients) such that
$p(x)=q(x)r(x)$. We also write $q(x)\mid p(x)$.
\begin{itemize}
  \item (Division algorithm) For any polynomial $q$ there are unique
    polynomials $f$ and $r$ such that $p(x)=q(x)f(x)+r$ and $\deg R<\deg q$.

    If $q$ has integer coefficients, then $f$ and $r$ have rational
    coefficients. If $q$ has a leading coefficient of $\pm 1$, then $f$ and $r$
    have integer coefficients.
  \item Since we proved Euclid's algorithm and B\'ezout's identity from the
    division algorithm, they still hold for polynomials.
  \item (Remainder theorem) If $p(a)=c$ then $x-a\mid p(x)-c$.
  \item (*) If $a$ and $b$ are integers, then $a-b\mid p(a)-p(b)$.
\end{itemize}
\section{Primitive polynomials}
An integer polynomial is \emph{primitive} if its coefficients have gcd $1$.
\begin{itemize}
  \item Every nonzero rational polynomial has exactly one primitive multiple
    with positive leading coefficient.
  \item (Gauss' Lemma) The product of two primitive polynomials is primitive.
  \item (Gauss' Lemma, alternate form) If an integer polynomial is the product
    of two nonconstant rational polynomials then it is the product of two
    nonconstant integer polynomials.
  \item (Rational Root Theorem) If $y$ and $z$ are integers with $\gcd(y,z)=1$
    such that $p(y/z)=0$ then $y\mid a_0$ and $z\mid a_n$.
\end{itemize}
\section{Irreducibility}
An integer polynomial is \emph{irreducible} over $\mathbb Z$ (for the rest of
this handout, I'll shorten this to irreducible) if it is not the product of two
nonconstant integer polynomials.

Usually, to prove irreducibility you will assume for contradiction that the
polynomial is reducible. Modular arithmetic arguments on the coefficients are
often useful.
\begin{itemize}
  \item (Eisenstein's Criterion) If there exists a prime $q$ such that $q^2\nmid
    a_0$, $q\mid a_i$ for each $i$ from $0$ to $n-1$, and $q\nmid a_n$, then $p$
    is irreducible.
\end{itemize}
\section{Problems}
\begin{enumerate}
  \item Prove that every nonconstant integer polynomial has a composite number
    in its image.
  \item (Schur's Theorem) Prove that for every nonconstant integer polynomial,
    the set of primes that divide some element of its image is infinite.
  \item Let $p$ be an integer polynomial and let $a$ be an integer such that
    $p(p(\cdots(p(a))\cdots))=a$. Prove that $p(p(a))=a$.
  \item Let $a,b,c$ be integers such that $a/b+b/c+c/a$ and $a/c+c/b+b/a$ are
    both integers. Prove that $|a|=|b|=|c|$.
  \item Prove that if $p$ is prime, then $1+x+x^2+\cdots+x^{p-1}$ is
    irreducible.
  \item Let $f$ be a nonconstant integer polynomial and let $n$ and $k$ be
    positive integers. Prove that there exists a positive integer $a$ such that
    each of the numbers $f(a),f(a+1),\ldots,f(a+n-1)$ has at least $k$ distinct
    prime divisors.
  \item Prove that if $5\nmid a$, then $x^5-x+a$ is irreducible.
  \item Find all integer polynomials $p$ such that
    \begin{itemize}
      \item $p(n)>n$ for all positive integers $n$, and
      \item for each positive integer $n$ there is a positive integer $k$ such
        that $p^{(k)}(1)$ ($p$ repeated $k$ times) is divisible by $n$.
    \end{itemize}
\end{enumerate}
\newpage
\section{Homework}
\begin{enumerate}
  \item Find all integer polynomials $p$ such that $n\mid p(2^n)$ for all
    positive integers $n$.
  \item Let $p$ be an irreducible integer polynomial. Prove that $p$ does not
    have multiple roots.
  \item Find all positive integers $k$ for which the following statement is
    true: if $p$ is an integer polynomial such that $0\le p(i)\le k$ for each
    integer $0\le i\le k+1$, then all of these $p(i)$s are equal.
\end{enumerate}
\end{document}
