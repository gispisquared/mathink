\documentclass{article}
\usepackage[inline]{asymptote}
\usepackage{enumitem,amsmath,amsfonts,amssymb,geometry,parskip}
\def\asydir{asy}
\title{Integer Polynomials}
\author{Andres Buritica Monroy}
\newcommand\lcm{\mathrm{lcm}}
\newcommand\Zz{\mathbb{Z}}
\newcommand\ord{\mathrm{ord}}
\date{}
\begin{document}
\maketitle
\section{Basic facts}
For all of these, let $p(x)=\sum_{i=0}^n a_i x^i$ be a polynomial with integer
coefficients.

We say that $p(x)$ is \emph{divisible} by a polynomial $q(x)$ if there is a
polynomial $r(x)$ (not necessarily with integer coefficients) such that
$p(x)=q(x)r(x)$. We also write $q(x)\mid p(x)$.
\begin{itemize}
	\item (Division algorithm) For any polynomial $q$ there are unique
	      polynomials $f$ and $r$ such that $p(x)=q(x)f(x)+r$ and $\deg r<\deg q$.

	      If $q$ has integer coefficients, then $f$ and $r$ have rational
	      coefficients. If $q$ has a leading coefficient of $\pm 1$, then $f$ and $r$
	      have integer coefficients.
	\item Since we proved Euclid's algorithm and B\'ezout's identity from the
	      division algorithm, they still hold for polynomials.
	\item (Remainder theorem) If $p(a)=c$ then $x-a\mid p(x)-c$.
	\item (Finite differences) If $c$ is a constant and $p$ is a polynomial with
	      leading coefficient $ax^n$, then $p(x)-p(x-c)$ is a polynomial with
	      leading coefficient $nax^{n-1}$.
	\item (*) If $a$ and $b$ are integers, then $a-b\mid p(a)-p(b)$.
\end{itemize}
\section{Primitive polynomials}
An integer polynomial is \emph{primitive} if its coefficients have gcd $1$.
\begin{itemize}
	\item Every nonzero rational polynomial has exactly one primitive multiple
	      with positive leading coefficient.
	\item (Gauss' Lemma) The product of two primitive polynomials is primitive.
	\item (Gauss' Lemma, alternate form) If an integer polynomial is the product
	      of two nonconstant rational polynomials then it is the product of two
	      nonconstant integer polynomials.
	\item (Rational Root Theorem) If $y$ and $z$ are integers with $\gcd(y,z)=1$
	      such that $p(y/z)=0$ then $y\mid a_0$ and $z\mid a_n$.
\end{itemize}
\section{Irreducibility}
An integer polynomial is \emph{irreducible} over $\mathbb Z$ (for the rest of
this handout, I'll shorten this to irreducible) if it is not the product of two
nonconstant integer polynomials.

Usually, to prove irreducibility you will assume for contradiction that the
polynomial is reducible. Modular arithmetic arguments on the coefficients are
often useful.
\begin{itemize}
	\item (Unique factorisation) Every integer polynomial can be factorised into a
	      product of a constant and primitive irreducible integer
	      polynomials. This factorisation is unique up to the permutation and sign of
	      these polynomials.
	\item (Eisenstein's Criterion) If there exists a prime $q$ such that $q^2\nmid
		      a_0$, $q\mid a_i$ for each $i$ from $0$ to $n-1$, and $q\nmid a_n$, then $p$
	      is irreducible.
	\item If an integer polynomial is irreducible mod $n$ for some positive integer
	      $n$, then it is irreducible.
\end{itemize}
\section{Polynomials mod $p$}
Let $p$ be prime.
\begin{itemize}
	\item Prove that unique factorisation holds for polynomials mod $p$. (This is
	      not true for all integers --- for instance,
	      $(x-1)^2\equiv(x-3)^2\pmod 4$.)
	\item Prove that for every function $f:\Zz_p\to\Zz_p$ there is a unique polynomial $P$ in
	      $\Zz_p$ of degree less than $p-1$ such that $f(x)=P(x)$ for each
	      $x\in\Zz_p$.
	\item Let $g$ be a generator mod $p$, and let $ab=p-1$. Prove that
	      \[\prod_{i=1}^a (x-g^{bi})\equiv x^a-1\pmod p.\]
	      What does this tell us about the roots of the cyclotomic polynomials in mod
	      $p$?
	\item Consider all $\binom{p-1}k$ products of $k$ elements of $\Zz_p$. Prove
	      that their sum is divisible by $p$.
	\item For any positive integer $n<p-1$, prove that
	      \[\sum_{i=1}^{p-1} i^n\equiv 0\pmod p.\]
\end{itemize}
\section{Problems}
\begin{enumerate}
	\item (Schur's Theorem) Prove that for every nonconstant integer polynomial,
	      the set of primes that divide some element of its image is infinite.
	\item Let $p$ be an integer polynomial and let $a$ be an integer such that
	      $p(p(\cdots(p(a))\cdots))=a$. Prove that $p(p(a))=a$.
	\item Let $p$ be a polynomial of dgree $d$. Find a linear
	      equation that the values $p(0),p(1),\ldots,p(d+1)$ always satisfy.
	\item Let $P(x)$ and $Q(x)$ be polynomials whose coefficients are all equal to
	      $1$ or $7$. If $P(x)$ divides $Q(x)$, prove that $1+\deg P(x)$ divides
	      $1+\deg Q(x)$.
	\item Let $p$ be an irreducible integer polynomial. Prove that $p$ does not
	      have multiple roots.
	\item Let $a,b,c$ be integers such that $a/b+b/c+c/a$ and $a/c+c/b+b/a$ are
	      both integers. Prove that $|a|=|b|=|c|$.
	\item Prove that if $p$ is prime, then $1+x+x^2+\cdots+x^{p-1}$ is
	      irreducible.
	\item Prove that if $5\nmid a$, then $x^5-x+a$ is irreducible.
	\item Find all integer polynomials $p$ such that
	      \begin{itemize}
		      \item $p(n)>n$ for all positive integers $n$, and
		      \item for each positive integer $n$ there is a positive integer $k$ such
		            that $p^{(k)}(1)$ ($p$ repeated $k$ times) is divisible by $n$.
	      \end{itemize}
\end{enumerate}
\newpage
\section{Homework}
\begin{enumerate}
	\item Find all integer polynomials $p$ such that $n\mid p(2^n)$ for all
	      positive integers $n$.
	\item Let $p$ be prime. Find the least residue of the product of $(4-x)$ mod
	      $p$, where $x$ runs over all residues mod $p$ except the quadratic
	      residues.
	\item Find all positive integers $k$ for which the following statement is
	      true: if $p$ is an integer polynomial such that $0\le p(i)\le k$ for each
	      integer $0\le i\le k+1$, then all of these $p(i)$s are equal.
\end{enumerate}
\end{document}
