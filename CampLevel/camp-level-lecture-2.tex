\documentclass{article}
\usepackage[inline]{asymptote}
\usepackage{enumitem,amsmath,amsfonts,amssymb,geometry,parskip}
\def\asydir{asy}
\title{$\mathbb Z_p$ and prime factorisations}
\author{Andres Buritica}
\date{April 23, 2022}
\begin{document}
\maketitle
\section{The integers modulo a prime}
  Let $p$ be a prime.

  We define $\mathbb Z_p$ (the integers mod $p$) using an equivalence relation
  \[a\equiv b\pmod p\iff p\mid b-a\]
  over the integers.

  This gives us $p$ equivalence classes corresponding to the least residues mod
  $p$: \[\{0,1,2,\ldots,p-1\}.\]
  \begin{itemize}
    \item Prove that addition, subtraction, multiplication and exponentiation
      are consistently defined: that
      is, if $a,b,c,d$ are integers with $a\equiv b\pmod p$ and $c\equiv d\pmod
      p$ then \[a+c\equiv b+d\pmod p,\quad a-c\equiv b-d\pmod p,\quad ac\equiv
        bd\pmod p,\quad a^n\equiv b^n\pmod p.\]
    \item Prove that for any integer $a$ with $p\nmid a$ there is an integer $b$
      with $0\le b<p$ such that $ab\equiv 1\pmod p$. We call $b$ the inverse of
      $a$ in mod $p$, notated $a^{-1}$.
    \item We may define fractions in mod $p$ as \[\frac ab\equiv ab^{-1}\pmod p,\]
      assuming $p\nmid b$. Prove that addition, subtraction, multiplication
      and division by anything nonzero still work as expected.
    \item Prove that $(p-1)!\equiv -1\pmod p$.
    \item Find the least residues of 
      \[\binom {p-1}k\quad\text{and}\quad\frac1p\binom pk\] in mod $p$.
    \item Let $\mathbb Z_p^*$ be the set of nonzero residues mod $p$, and let $a$ be an
      element of $\mathbb Z_p^*$. Prove that the function $f(x)=ax$ is a
      bijection from $\mathbb Z_p^*$ to $\mathbb Z_p^*$. Deduce that $p\mid
      a^{p-1}-1$.
    \item Prove that if $a^m\equiv 1\pmod p$ and $a^n\equiv 1\pmod p$ then
      $a^{\gcd(m,n)}\equiv 1\pmod p$.
  \end{itemize}
  \newpage
\section{Arithmetic functions}
  We define:
  \begin{itemize}
    \item The number of positive divisors function $d(n)$.
    \item The sum of positive divisors function $\sigma(n)$.
    \item The totient function $\varphi(n)$: the number of positive integers
      which are at most $n$ and coprime to $n$.
  \end{itemize}
  A function $f:\mathbb N\to\mathbb R$ is called multiplicative if for any
  coprime positive integers $a$ and $b$, we have
  \[f(a)f(b)=f(ab).\]
  It's called completely multiplicative if this equation holds for \emph{any}
  positive integers $a$ and $b$
  \begin{itemize}
    \item Prove that the values at the primes of a completely multiplicative
      function completely define the function (unless these values are all 0, in
      which case $f(1)$ can be 0 or 1).
    \item Prove that the values at prime powers of a multiplicative function
      completely define it (once again, unless these values are all 0).
    \item Prove that $d$ and $\sigma$ are multiplicative.
    \item
      Find formulae for $d(n),\sigma(n),\varphi(n)$ where
      $n=p_1^{e_1}p_2^{e_2}\cdots p_k^{e_k}$.
  \end{itemize}
\section{Problems}
\begin{enumerate}
  \item Prove that if there are two terms of an arithmetic progression which are
    coprime integers, then there is an infinite subset of that
    progression all of whose elements are coprime integers.
  \item Let $n$ be an even positive integer such that $\sigma(n)=2n$. Prove that
    $n=2^{p-1}\left(2^p-1\right)$, where $p$ is a prime.
  \item Define the sequence $a_n=2^n+3^n+6^n-1,\ n\in\mathbb N$.
    Find all primes which do not divide $a_n$ for any $n$.
  \item For any positive integer $n$, prove that $\displaystyle\sum_{d\mid
    n}\varphi(d)=n$.
  \item Let $x$ and $y$ be positive integers and let $p$ be prime. Assume there
    are coprime positive integers $m$ and $n$ such that $x^m\equiv y^n\pmod p$.
    Prove that there is a unique positive integer $z$ with $0\le z<p$ such that
    \[x\equiv z^n\pmod p,\qquad y\equiv z^m\pmod p.\]
  \item Let $a$ and $b$ be positive integers such that $a^n+n\mid b^n+n$ for all
    positive integers $n$. Prove that $a=b$.
  \item Let $n$ and $k$ be positive integers such that $\varphi^k(n)=1$ (that
    is, $\varphi$ iterated $k$ times). Prove that
    $n\le 3^k$.
\end{enumerate}
\newpage
\section{Homework}
\begin{enumerate}
  \item Prove that $\sigma(n)<n\sqrt{2d(n)}$ for all positive integers $n$.
  \item Given a positive integer $k$, show that there exists a prime $p$ such
    that one can choose distinct integers
    $a_1,a_2,\ldots,a_{k+3}\in\{1,2,\ldots,p-1\}$ such that $p$ divides
    $a_i a_{i+1}a_{i+2}a_{i+3}-i$ for all $i=1,2,\ldots,k$.
  \item Find all completely multiplicative functions $f:\mathbb N\to\mathbb R$
    such that for all $a,b\in\mathbb N$, at least two of $f(a),f(b),f(a+b)$ are
    equal.
\end{enumerate}
\end{document}
