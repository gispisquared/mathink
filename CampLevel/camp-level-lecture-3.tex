\documentclass{article}
\usepackage[inline]{asymptote}
\usepackage{enumitem,amsmath,amsfonts,amssymb,geometry,parskip}
\def\asydir{asy}
\newcommand\lcm{\mathrm{lcm}}
\title{$\mathbb Z_n$ and Diophantine equations}
\author{Andres Buritica Monroy}
\date{}
\begin{document}
\maketitle
\section{The integers modulo an integer}
Let $n$ be a positive integer.

We define $\mathbb Z_n$ (the integers mod $n$) using an equivalence relation
\[a\equiv b\pmod n\iff n\mid b-a\]
over the integers.

This gives us $n$ equivalence classes corresponding to the least residues mod
$n$: \[\{0,1,2,\ldots,n-1\}.\]
\begin{itemize}
	\item Prove that addition, subtraction, multiplication and exponentiation
	      are consistently defined: that
	      is, if $a,b,c,d$ are integers with $a\equiv b\pmod n$ and $c\equiv d\pmod
		      n$ then \[a+c\equiv b+d\pmod n,\quad a-c\equiv b-d\pmod n,\quad ac\equiv
		      bd\pmod n,\quad a^m\equiv b^m\pmod n.\]
	\item Prove that for any integer $a$ with $\gcd(n,a)=1$ there is an integer $b$
	      with $0\le b<n$ such that $ab\equiv 1\pmod n$. We call $b$ the inverse of
	      $a$ in mod $n$, notated $a^{-1}$.
	\item We may define fractions in mod $n$ as \[\frac ab\equiv ab^{-1}\pmod n,\]
	      assuming $\gcd(b,n)=1$. Prove that addition, subtraction, and
	      multiplication still work.
	\item Let's say you have integers $a,b,c,n$ such that $ab\equiv ac\pmod n$.
	      What can you say about $c-b$?
	\item Let $\mathbb Z_n^*$ be the set of nonzero residues mod $n$ that are
	      coprime to $n$, and let $a$ be an
	      element of $\mathbb Z_n^*$. Prove that the function $f(x)=ax$ is a
	      bijection from $\mathbb Z_n^*$ to $\mathbb Z_n^*$. Deduce that $n\mid
		      a^{\varphi(n)}-1$. (Euler's Theorem. The case where $n$ is prime and
	      $\varphi(n)=n-1$ is a special case known as Fermat's Little Theorem.)
	\item Prove that if $a^x\equiv 1\pmod n$ and $a^y\equiv 1\pmod n$ then
	      $a^{\gcd(x,y)}\equiv 1\pmod n$. (GCD Trick)
	\item Let $m$ and $n$ be coprime positive integers. For any integers $a$ and
	      $b$, prove that there is a unique residue $c$ mod $mn$ such that $a\equiv
		      c\pmod m,\ b\equiv c\pmod n$. (Special case of Chinese Remainder Theorem)
	\item Prove that $\varphi$ is multiplicative.
	\item What is the product of the elements of $\mathbb Z_n^*$ mod $n$?
\end{itemize}
\newpage
\section{Choosing good mods}
Prove that:
\begin{itemize}
	\item Squares are 0, 1 or 4 mod each of \{5,8\}, and 0 or 1 mod 3.
	\item Cubes are 0, 1 or $-1$ mod each of \{7,9\}.
\end{itemize}
In general, for $n$th powers, try looking mod $m$ where $\varphi(m)$ is a
small multiple of $n$.

Also, of course, try choosing a mod which divides a bunch of terms.
\section{Diophantine equation tricks}
\begin{itemize}
	\item Factorising expressions
	\item Using mods to find contradictions or get conditions on the variables
	\item Choosing a prime that divides some number or expression
	\item Reducing expressions mod other expressions
	\item Quadratic discriminant trick: if $a,b,c,n$ are positive integers such
	      that $an^2+bn+c=0$ then $b^2-4ac$ is a perfect square.
	\item (for later lectures) Bounding arguments, descent, $\nu_p$
	      considerations
\end{itemize}
\section{Problems}
\begin{enumerate}
	\item Find the minimum possible value of $m+n$, where $m$ and $n$ are distinct
	      positive integers such that $1000\mid 1978^m-1978^n$.
	\item Show that for any fixed integers $n$ and $a$, the sequence
	      $a,a^a,a^{a^a},\ldots$ is eventually constant mod $n$.
	\item An infinite arithmetic progression contains a perfect $a$th power and a
	      perfect $b$th power. Prove that it contains a perfect $\lcm(a,b)$th power.
	\item Prove that if $a$ and $b$ are positive integers, then $4ab-a-b$ is not a
	      perfect square.
	\item Let $n$ be a positive integer, and let $S$ be a set of $n$ positive
	      integers all at most $n^2$. Prove that there is a set $T$ of $n$ positive
	      integers such that the set $\{s+t:s\in S,t\in T\}$ covers at least half of
	      the residues mod $n^2$.
	\item Let $n$ and $z$ be integers greater than $1$ such that $\gcd(n,z)=1$.
	      Prove that there is some nonnegative integer $i<n$ such that
	      $1+z+z^2+\cdots+z^i$ is divisible by $n$.
	\item Find all positive integer solutions to $3^x+4^y=5^z$.
	\item Let $n>1$ be a positive integer and let $p$ be a prime. Given that
	      $n\mid p-1$ and $p\mid n^3-1$, prove that $4p-3$ is a perfect square.
\end{enumerate}
\newpage
\section{Homework}
\begin{enumerate}
	\item Let $n>1$ be an odd positive integer and let $S$ be the set of integers
	      $x$, with $1\le x\le n$, such that both $x$ and $x+1$ are coprime to $n$.
	      Find the product of the elements of $S$ mod $n$.
	\item What is the smallest positive integer $n$ for which there exist positive
	      integers $x_1,x_2,\ldots,x_n$ such that
	      \[x_1^3+x_2^3+\cdots+x_n^3=2002^{2002}?\]
	\item Find all integers $x,y$ such that $(x^2+y)(x+y^2)=(x-y)^3$.
\end{enumerate}
\end{document}
