\documentclass{article}
\usepackage[inline]{asymptote}
\usepackage{enumitem,amsmath,amsfonts,geometry,parskip,amssymb}
\def\asydir{asy}
\title{Descent, Vieta Jumping, and Pell equations}
\author{Andres Buritica Monroy}
\date{}
\begin{document}
\maketitle
\section{Techniques}
Vieta jumping and Pell equations are two special cases of the technique of
\emph{infinite descent}. As we know, this technique involves assigning some
notion of \emph{size} to every set of positive integers satisfying a certain
property, and then proving that if a certain condition is not satisfied, one can
always get from one set to another set of smaller size.

In Vieta jumping, the way we get from one solution to the next is by considering
the second root of a quadratic which we know has one integer root; then, Vieta's
formulas provide us with information that allows us to prove the new solution
has smaller size.

In Pell equations, the key step is provided by \emph{Brahmagupta's
Identity}:
\[(a^2-nb^2)(c^2-nd^2)=(ac\pm nbd)^2-n(ad\pm bc)^2.\]
\section{Pell Equations}
We consider the equation $a^2-nb^2=\pm 1$ over positive integers $a,b$, where
$n$ is not a square. In homework question 3, you will prove that a solution
always exists.

If we consider a solution $(a_0,b_0)$, then we may find more
solutions by repeatedly applying Brahmagupta's Identity. More explicitly, if
$(a,b)$ is a solution then we get the bigger solution $(a_0 a+nb_0 b,a_0 b+b_0
a)$.

Note that if $a_0^2-nb_0^2=1$, then all solutions thus generated satisfy
$a^2-nb^2=1$. Otherwise, the RHSs alternate between $1$ and $-1$.

I claim that this process yields all solutions, assuming we begin with the
smallest solution $(a_0,b_0)$. Consider the descent step
\[(a,b)\to (|a_0 a-nb_0 b|,|a_0 b-b_0 a|).\]
\begin{itemize}
    \item Prove that this step inverts the previous type of step.
    \item Prove that this step gets us from a solution to a smaller solution,
        unless $(a,b)=(a_0,b_0)$.
\end{itemize}

When we instead have an equation of the form $a^2-nb^2=\pm k$, we may use the
same descent step. The descent
can only finish when $a_0 b-b_0 a=0$ or $|a_0 b-b_0 a|\ge b$. In the second case 
we may deduce
$b^2<\frac{b_0^2|k|}{2(a_0-1)}$,
which leaves us with a finite set of solutions from which all solutions are
generated.
\newpage
\section{Problems}
\begin{enumerate}
  \item Let $a$ and $b$ be positive integers.
    Prove that if \(\frac{a^2+b^2}{ab+1}\) is an integer, then it is a square.
  \item Prove that infinitely many triangular numbers are squares.
  \item Find all solutions in integers to $x^2+y^2+z^2=2xyz$.
  \item Prove that there are infinitely many triples $(a,b,c)$ of positive
    integers in arithmetic progression
    such that $ab+1$, $bc+1$ and $ca+1$ are all perfect squares.
  \item Prove that there are infinitely many pairs of positive integers $a,b$
    such that $a\mid b^2+1$ and $b\mid a^2+1$.
\item Prove that if $2+\sqrt{28n^2+1}$ is an integer, then it is a perfect
    square.
  \item Let $a$ and $b$ be two positive integers. Prove that
    \[a^2+\left\lceil\frac{4a^2}b\right\rceil\] is not a square.
\end{enumerate}
\newpage
\section{Homework}
  \begin{enumerate}
    \item Find all positive integers $n$ such that \[\sqrt{\frac{7^n+1}2}\] is
      prime.
    \item Let $a$ and $b$ be positive integers. Show that if $4ab-1$ divides
      $(4a^2-1)^2$, then $a=b$.
    \item Let $n$ be a positive integer.
    \begin{enumerate}
        \item Prove that there are infinitely many pairs $(a,b)$ of positive
            integers such that
            \[
                \left|b\sqrt n-a\right|<\frac 1b.
            \]
        \item Prove that there are infinitely many pairs $(a,b)$ of positive
            integers such that $a^2-nb^2$ equals the same value for each such pair.
        \item Prove that there are positive integers $a_1,b_1,a_2,b_2$
            such that $a_1^2-nb_1^2=a_2^2-nb_2^2=d$, $a_1\equiv a_2\pmod d$, and
            $b_1\equiv b_2\pmod d$.
        \item Prove that there are positive integers $a_0,b_0$ such that
            $a_0^2-nb_0^2=1$.
    \end{enumerate}
  \end{enumerate}
\end{document}
