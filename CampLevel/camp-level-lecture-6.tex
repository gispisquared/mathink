\documentclass{article}
\usepackage[inline]{asymptote}
\usepackage{enumitem,amsmath,amsfonts,geometry,parskip,amssymb}
\def\asydir{asy}
\title{Density Arguments, Floor Functions}
\author{Andres Buritica Monroy}
\date{}
\begin{document}
\maketitle
\section{Techniques}
  Density arguments: these involve taking a ``global'' view that involves
  estimating how many numbers have a specific property in a specific range.
  The aim is usually to apply pigeonhole.

  Floor functions: recall $\lfloor x\rfloor$ is the largest integer which is at
  most $x$. Most often, problems which involve $\lfloor x\rfloor$ will be
  solved by considering $x-\lfloor x\rfloor$ (also denoted $\{x\}$), or $\lceil
  x\rceil-x$.
\section{Problems}
\begin{enumerate}
  \item Is there a positive integer which can be written as the sum of 2023
    distinct 2022nd powers in at least 2021 different ways?
  \item Prove that the sequence $a_i=\left\lfloor(\sqrt 2+1)^i\right\rfloor$
    alternates between even and odd integers.
  \item Let $r$ be an irrational root of a polynomial $P(x)$ of degree $d$ with
    integer coefficients. Prove that there is a real number $C$ such that for
    any integer $q$ we have $\{qr\}\ge \frac C{q^{d-1}}$.
  \item Define
    \[q(n)=\left\lfloor\frac n{\lfloor\sqrt n\rfloor}\right\rfloor.\]
    Determine all positive integers $n$ for which $q(n)>q(n+1)$.
  \item Let $P$ be a polynomial of degree larger than $1$ with integer
    coefficients. Prove that there are infinitely many positive integers which cannot be
    written in the form $P(x+1)+P(x+2)+\cdots+P(x+k)$ for positive integers $x$
    and $k$.
  \item Let $x$ be an irrational number. Prove that there are infinitely many
    positive integers $n$ such that $\{nx\}<\frac 1n$.
  \item Prove that for some constant $C>0$, the following statement holds:

    Let $m\ge 2$ be an integer, $A$ a finite set of integers (not
    necessarily positive), and $B_1,B_2,\ldots,B_m$ subsets of $A$. Suppose
    that for every $k=1,2,\ldots,m$, the sum of the elements of $B_k$ is
    $2^k$. Then $A$ contains at least $\frac {Cm}{\log_2 m}$ elements.
\end{enumerate}
\newpage
\section{Homework}
  \begin{enumerate}
    \item Determine all positive integers $M$ such that the sequence $a_0, a_1,
      a_2, \ldots$ defined by \[ a_0 = M + \frac{1}{2} \qquad \textrm{and}
        \qquad a_{k+1} = a_k\lfloor a_k \rfloor \quad \textrm{for} \, k = 0, 1,
      2, \ldots \]contains at least one integer term.
    \item Let $n>1$ be an integer, and let $a$ be an integer coprime to $n$. Prove
      that there exist integers $x,y$ with $0<|x|<\sqrt n$, $0<|y|<\sqrt n$ and
      $ay\equiv x\pmod n$.
    \item Prove that every positive integer is the root of a polynomial all of whose
      coefficients are of the form $2^a-2^b$ for positive integers $a$ and $b$.
  \end{enumerate}
\end{document}
