\documentclass{article}
\usepackage[inline]{asymptote}
\usepackage{enumitem,amsmath,amsfonts,geometry,parskip,amssymb}
\def\asydir{asy}
\newcommand\lcm{\mathrm{lcm}}
\title{$\nu_p$ notation and LTE}
\author{Andres Buritica Monroy}
\date{}
\begin{document}
\maketitle
\section{Definition}
Let $p$ be a prime and let $r$ be a rational number. We let $\nu_p(r)$ be the
exponent of $p$ in the prime factorisation of $r$.
\section{Basic properties}
\begin{itemize}
	\item $\nu_p(ab)=\nu_p(a)+\nu_p(b)$.
	\item $\nu_p(a+b)\ge\min(\nu_p(a),\nu_p(b))$.
	\item $\nu_p(a^n)=n\nu_p(a)$.
	\item If $a$ is a positive integer, then $\nu_p(a)\le\log_p(a)$.
	\item If $\nu_p(a+b)>\min(\nu_p(a),\nu_p(b))$ then $\nu_p(a)=\nu_p(b)$.
	\item $\frac ab$ is an integer iff for all $p$, $\nu_p(a)\ge\nu_p(b)$.
	\item $a$ is a perfect $k$th power iff for all $p$, $k\mid\nu_p(a)$.
	\item $\nu_p(\gcd(a,b))=\min(\nu_p(a),\nu_p(b))$.
	\item $\nu_p(\lcm(a,b))=\max(\nu_p(a),\nu_p(b))$.
\end{itemize}
\section{Less basic properties}
\begin{itemize}
	\item Legendre's Formula: (Here, $s_p(n)$ is the sum of the digits of $n$ in
	      base $p$.)
	      \[\nu_p(n!)=\sum_{i=1}^\infty\left\lfloor\frac
		      n{p^i}\right\rfloor=\frac{n-s_p(n)}{p-1}<\frac
		      n{p-1}.\]
	\item Lifting the Exponent (LTE):

	      Let $p$ be an odd prime, and let $a$ and $b$ be integers such that $p\mid
		      a-b$ but $p\nmid a$. Let $k$ be a positive integer.

	      Then, $\nu_p\left(a^k-b^k\right)=\nu_p(a-b)+\nu_p(k)$.
	\item LTE for $p=2$:

	      Let $a$ and $b$ be odd integers, and let $k$ be a positive integer.

	      If $k$ is even, we have $\nu_2(a^k-b^k)=\nu_2(a-b)+\nu_2(a+b)+\nu_2(k)-1$.

	      If $k$ is odd, we have $\nu_2(a^k-b^k)=\nu_2(a-b)$.
\end{itemize}
\section{Problems}
\begin{enumerate}
	\item Let $x,y,z$ be rational numbers such that $xy,\ yz,\ zx,\ x+y+z$ are all
	      integers.

	      Prove that $x,\ y,\ z$ are all integers.
	\item Let $p$ be an odd prime. Prove that
	      \[p^2\mid 1^p+2^p+\cdots+p^p.\]
	\item Let $a,b,c$ be positive integers such that $c\mid a^c-b^c$. Prove that
	      $c(a-b)\mid a^c-b^c$.
	\item Prove that for all positive integers $n$,
	      \[\binom{2n}n\mid\lcm(1,2,\ldots,2n).\]
	\item Find all pairs of positive integers $x,p$ such that $p$ is prime, $x\le
		      2p$, and $x^{p-1}\mid (p-1)^x+1$.
	\item Let $a,b,n$ be positive integers such that $a>b>1$ and $b$ is odd. If
	      $b^n\mid a^n-1$, prove that $na^b>3^n$.
	\item Find all natural numbers $n$ such that $n^2\mid 2^n+1$.
\end{enumerate}
\newpage
\section{Homework}
\begin{enumerate}
	\item Let $a$ be a positive integer such that $4(a^n+1)$ is a perfect
	      cube for all positive integers $n$. Prove that $a=1$.
	\item Let $n$ and $k$ be positive integers. Assume that for each positive
	      integer $m$, there exists a positive integer $a$ such that $a^k\equiv
		      n\pmod m$. Prove that $n$ is a perfect $k$th power.
	\item Find all positive integers $n$ and $k$ such that
	      \[k!=(2^n-1)(2^n-2)(2^n-4)\cdots(2^n-2^{n-1}).\]
\end{enumerate}
\end{document}
