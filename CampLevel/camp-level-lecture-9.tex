\documentclass{article}
\usepackage[inline]{asymptote}
\usepackage{enumitem,amsmath,amsfonts,geometry,parskip,amssymb}
\def\asydir{asy}
\newcommand\lcm{\mathrm{lcm}}
\newcommand\Zz{\mathbb{Z}}
\newcommand\ord{\mathrm{ord}}
\title{Hensel's Lemma, quadratic residues}
\author{Andres Buritica Monroy}
\date{}
\begin{document}
\maketitle
Once again, $p$ is a prime and $n$ is a positive integer throughout.
\section{Hensel's Lemma}
Let
\[P(x)=a_0x^0+\cdots+a_n x^n\] be a polynomial with integer coefficients.
We define the \emph{derivative}
\[P'(x)=a_1 x^0+2a_2 x^1+3a_3 x^3+\cdots+na_n x^{n-1}.\]
Let $r$ be an integer such that $P(r)\equiv 0\pmod n$ but $\gcd(P'(r),n)=1$.
Prove that for any positive integer $m$, there is a unique $s$ mod $n^m$ such
that $s\equiv r\pmod n$ and $P(s)\equiv 0\pmod {n^m}$.

The case where $n=p$ is prime is most common. In this case, the condition
$P(r)\equiv 0\pmod p$ and $P'(r)\not\equiv 0\pmod p$ is equivalent to $r$ being
a single root of $P$ mod $p$. That is, $(x-r)\mid P(x)$ but $(x-r)^2\nmid
P(x)$.
\section{Quadratic residues}
If $x$ can be written as $y^2$ for $y\in\Zz_n^*$, then we say that $x$ is a
\emph{quadratic residue (QR)} mod $n$. Note that $0$ is not a QR mod $n$.

Prove that $x$ is a QR mod $n$ iff both
\begin{itemize}
  \item $x\equiv 1\pmod{\gcd(8,n)}$, and
  \item for each odd prime $p\mid n$, $x$ is a QR mod $p$.
\end{itemize}

Hence, we now restrict ourselves to considering QRs mod $p$.
Define the \emph{Jacobi symbol}
\[\left(\frac ap\right)=\begin{cases} 0&p\mid a \\ 1&a\text{ is a QR mod }p \\
-1&\text{otherwise}\end{cases}.\]

\begin{itemize}
  \item (Euler's criterion) Prove that
    $\left(\frac ap\right)\equiv a^{\frac{p-1}2}\pmod p$.
    Hence, $\left(\frac ap\right)\left(\frac
    bp\right)=\left(\frac{ab}p\right)$.
  \item (Gauss' Lemma) Let $a\in\Zz_p^*$, and let $S\subseteq\Zz_p^*$ such that $x\in S\iff
    -x\not\in S$. Let $T=\{ay:y\in S\}$. Then
    $\left(\frac ap\right)=(-1)^{|T\setminus S|}$.
  \item Find $\left(\frac 2p\right)$ and $\left(\frac{-1}p\right)$.
\end{itemize}

Finally, there is quadratic reciprocity which we won't prove today.
Let $p$ and $q$ be distinct odd primes. Then,
\[\left(\frac qp\right)\left(\frac pq\right)=(-1)^{\frac{(p-1)(q-1)}4}.\]
\section{Problems}
\begin{enumerate}
  \item Let $p$ be an odd prime and let $a$ be an integer with $\gcd(a,p)=1$.
    Prove that
    \[\sum_{n=1}^p\left(\frac{n^2+a}p\right)=-1.\]
  \item Let $k$ and $n$ be positive integers such that $n$ is odd. Prove that
    there is an integer $a$ such that $a^{32}\equiv(n+1)^3\pmod{n^k}$.
  \item Prove that for any prime $p$ and positive integer $a$ with $p\nmid a$
    there are at least $p-1$ solutions in $\Zz_p$ to $x^2+y^2\equiv a\pmod p$.
  \item Let $p$ be prime and let $a$ and $b$ be positive integers such that
    $p\nmid b$. Prove that there exists a positive integer $n$ such that
    $p^a\mid n^n-b$.
  \item If $p>3$ is a prime such that $\varphi(p-1)>\frac{p-1}3$, prove that
    there are two consecutive generators mod $p$.
  \item Let $P$ be a nonconstant polynomial with integer coefficients. Prove that for any
    integer $m$ there exist an integer $n$ and a prime $p$ such that $p^m\mid
    P(n)$.
  \item Let $a_1,a_2,a_3,\ldots$ be a sequence of integers, such that for any
    positive integers $n$ and $k$, the quantity
    \[\frac{a_n+a_{n+1}+\cdots+a_{n+k-1}}k\]
    is always the square of an integer. Prove that all $a_i$s are equal.
  \item Find all completely multiplicative functions $f:\mathbb N\to\mathbb Z$
    such that for all $a,b\in\mathbb N$, at least two of $f(a),f(b),f(a+b)$ are
    equal.
\end{enumerate}
\newpage
\section{Homework}
\begin{enumerate}
  \item Let $k=2^{2^n}+1$ for some positive integer $n$. Prove that $k$ is prime
    if and only if $k\mid 3^{(k-1)/2}+1$.
  \item Find all positive integers $k$ such that for all positive integers $n$,
    there exist a prime $p$ and positive integers $x$ and $y$ for which
    $\gcd(x,y)=1$ and $p^n\mid \frac{x^k-y^k}{x-y}$.
  \item Let $p>3$ be a prime and let $a,b,c$ be integers. Suppose
    that $ax^2+bx+c$ is a perfect square for $p$ consecutive integers $x$. Prove
    that $p\mid b^2-4ac$.
\end{enumerate}
\end{document}
