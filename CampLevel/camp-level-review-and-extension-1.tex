\documentclass{article}
\usepackage[inline]{asymptote}
\usepackage{enumitem,amsmath,amsfonts,amssymb,geometry,parskip}
\def\asydir{asy}
\title{Review and Extension --- Divisibility and Congruences}
\author{Andres Buritica}
\date{May 14, 2022}
\begin{document}
\maketitle
\section{Exam Review}
\section{Key concepts for this term}
  \begin{itemize}
    \item Induction, strong induction, well-ordering
    \item If $a\mid b$ and $a\mid c$ then for any integers $x$ and $y$, $a\mid
      bx+cy$
    \item Division algorithm, Euclid's algorithm, Bezout's identity
    \item Fundamental Theorem of Arithmetic
    \item GCD, LCM in terms of prime factorisations
    \item Factorise expressions
    \item Take out the GCD
    \item Basic properties of $\mathbb Z_p$: operations, inverses, Wilson,
      Fermat, GCD trick
    \item Multiplicative and completely multiplicative functions
    \item Formulas for $d$, $\sigma$ and $\varphi$
    \item Prove a problem for prime powers first
    \item Basic properties of $\mathbb Z_n$: operations, inverses, 
      Euler, GCD trick, Chinese Remainder Theorem, generalised Wilson
    \item How to choose a mod
    \item Modular contradictions
    \item Quadratic discriminant trick
  \end{itemize}
\section{Problems}
\section{Homework}
  Solve and submit any three problems from Section 3 and/or Section 7.
\newpage
\section{Extension: Farey sequences}
  Let $n$ be a fixed positive integer. Let
  $\frac{a_1}{b_1},\ldots,\frac{a_k}{b_k}$ be the rational numbers between 0
  and 1 inclusive with denominators at most $n$, written in increasing order
  and lowest terms.
  \begin{itemize}
    \item Prove that for each $i$, $a_{i+1}b_i-a_i b_{i+1}=1$.
    \item Prove that the rational number $x$ with smallest denominator such
      that $\frac{a_i}{b_i}<x<\frac{a_{i+1}}{b_{i+1}}$ is
      $\frac{a_i+a_{i+1}}{b_i+b_{i+1}}$. 
    \item Which pairs of numbers appear as consecutive $b_i$s?
  \end{itemize}
\section{Extension: Dirichlet Convolution and Mobius Inversion}
  Let $f:\mathbb N\to\mathbb R$ and $g:\mathbb N\to\mathbb R$ be two functions.
  We define the \emph{Dirichlet convolution} $f*g$ as
  \[(f*g)(n)=\sum_{d\mid n}f(d)g\left(\frac nd\right).\]

  We define the functions $d,\ \sigma,\ \varphi$ as before and also define the
  functions \[\zeta(n)=1,\ \psi(n)=n.\]
  \begin{itemize}
    \item Prove that $*$ is associative: that is,
      $(a*b)*c=a*(b*c)$.
    \item Prove that if $a$ and $b$ are multiplicative then so is $a*b$.
    \item Find a function $\delta$ such that $\delta*a=a$ for all functions $a$.
    \item Find a function $\mu$ such that $\mu*\zeta=\delta$.
    \item Prove that $g=f*\zeta\iff f=g*\mu$.
    \item Find $\zeta*\zeta,\ \psi*\zeta$ and $\varphi*\zeta$.
    \item Prove that
      \[\sum_{i=1}^n f(i)\left\lfloor\frac
          ni\right\rfloor=\sum_{j=1}^n(f*\zeta)(j).\]
  \end{itemize}
  \newpage
\section{Extension: Problems}
\begin{enumerate}
  \item Suppose that $(a_1,b_1),(a_2,b_2),\ldots,(a_{100},b_{100})$ are distinct
    ordered pairs of nonnegative integers.
    Let $N$ denote the number of pairs of integers $(i,j)$ satisfying
    $1\leq i<j\leq 100$ and $|a_i b_j-a_j b_i|=1$. 
    Determine the largest possible value of $N$ over all possible choices of the
    100 ordered pairs.
  \item For a positive integer $n$, let $f(n)$ be the number of binary strings of length $n$ that
      can't be expressed as an $m$-fold repetition of another binary string for
      any $m>1$.

      For example, $f(6)=54$ since the only strings of length 6 that can be
      expressed as an $m$-fold repetition of another binary string for some $m>1$
      are 000000, 001001, 010010, 010101, 011011, 100100, 101010, 101101, 110110,
      111111.
       
      \begin{enumerate}
        \item Find two functions $g$ and $h$, in closed form, such that $f=g*h$.
        \item 
          Prove that $n\mid f(n)$.
        \item Find all $n$ for which $n\mid\displaystyle\sum_{i=1}^n f(i)\left\lfloor\frac
              ni\right\rfloor$.
      \end{enumerate}
\end{enumerate}
\end{document}
