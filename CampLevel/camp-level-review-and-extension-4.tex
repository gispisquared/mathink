\documentclass{article}
\usepackage[inline]{asymptote}
\usepackage{enumitem,amsmath,amsfonts,geometry,parskip,amssymb,hyperref}
\hypersetup{colorlinks=true}
\def\asydir{asy}
\newcommand\lcm{\mathrm{lcm}}
\newcommand\Zz{\mathbb{Z}}
\newcommand\Nn{\mathbb{N}}
\newcommand\ord{\mathrm{ord}}
\title{Review and Extension: Constructions, Polynomials, Sequences}
\author{Andres Buritica Monroy}
\date{}
\begin{document}
\maketitle
\section{Key concepts for this term}
\begin{itemize}
	\item Existence results: Bezout, CRT, Pell equations, Hensel's
	      Lemma, Dirichlet, Bertrand, Fermat Christmas, Schur, Zsigmondy
	\item Control properties of constructions
	\item Division algorithm, Euclid and Bezout for polynomials
	\item Unique factorisation for polynomials with integer coefficients and
	      polynomials mod $p$
	\item Finite differences
	\item If $a$ and $b$ are integers, then $a-b\mid p(a)-p(b)$
	\item Primitive polynomials
	\item Gauss' Lemma
	\item Irreducibility and Eisenstein
	\item Sequences and integer functions
\end{itemize}
\section{What Now?}
You now know all of the number theory that you need to solve problems
at the high school olympiad level. What remains now
is to familiarise yourself with this toolkit and how each of the tools within it
can be applied in different situations. For this, there is no substitute for
practice.

Apart from the problems on these handouts, the best place to go for
practice problems is AoPS, especially
the pages for \href{https://artofproblemsolving.com/community/c14}{international
	contests} and \href{https://artofproblemsolving.com/community/c59}{TSTs}.
However, note that solutions on AoPS are user-contributed and often incorrect.
\end{document}
