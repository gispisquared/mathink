\documentclass{article}
\usepackage[inline]{asymptote}
\usepackage{enumitem,amsmath,amsfonts,geometry,parskip,amssymb,hyperref}
\def\asydir{asy}
\title{S-L Number Theory Course Information}
\author{Andres Buritica Monroy \\
\href{mailto:aburitica1729@gmail.com}{aburitica1729@gmail.com}}
\date{}
\begin{document}
\maketitle
Welcome to the Mathink IOD Senior-Level course! I will be your instructor for the
Number Theory portion. This class will cover the number theoretical knowledge
assumed by problem writers at competitions like the AMO, as well as giving you
tips about how to think in order to maximise your chances of solving problems.

In order to maximise the value you gain from this class, make sure to be an
active participant in lectures. This means not only taking notes, but also
asking questions if something is unclear and doing your best to answer any
questions I ask of the class.

Each handout will also include more problems than we can get to in class. These
problems are there for you to attempt both during and after the course as you
continue your Olympiad journey.
\section{Course Format}
The S-L Number Theory course is divided into four terms, each of which is five
weeks long. On the first three Sundays of each term I will teach a lecture. The
fourth Sunday of each term will be an exam joint with another Mathink course,
while on the fifth Sunday I will hold a review and extension session about the
term's material. Each week there will also be a student-run discussion session.

Homework is an essential part of this course. At the end of each lecture and
review session I will release three homework problems. You should spend at
least 15 minutes attempting each homework problem, and then neatly write up any
progress or solutions you find. Thus, you should be spending 1--2 hours a week
on number theory homework each week it is assigned.

Please submit both neat writeups and rough work
for every problem by the discussion session following the lecture. In the
discussion session, you should present any solutions you find to the homework.

Homework will be marked on the Google Drive (each problem is out of 7 points)
by the following lecture. Please
take the time to look at my comments, especially if you think you solved a
problem but did not get a 7 for it.
\newpage
\section{Syllabus}
Here is an outline of the topics planned for this year.
\begin{itemize}
  \item Term 1: Induction and Divisibility
    \begin{itemize}
      \item Lecture 1: Induction and divisibility basics
      \item Lecture 2: Division algorithm and applications
      \item Lecture 3: Prime factorisations
    \end{itemize}
  \item Term 2: Modular arithmetic
    \begin{itemize}
      \item Lecture 4: Basic definitions, existence of inverses
      \item Lecture 5: Fermat, Wilson, GCD trick
      \item Lecture 6: Euler, CRT, modular contradictions
    \end{itemize}
  \item Term 3: Diophantine equations
    \begin{itemize}
      \item Lecture 7: Bounding arguments
      \item Lecture 8: Infinite descent, floor functions
      \item Lecture 9: Assorted problems
    \end{itemize}
  \item Term 4: Other problem types
    \begin{itemize}
      \item Lecture 10: Constructions and existence proofs
      \item Lecture 11: Sequences and integer functions
      \item Lecture 12: Integer polynomials
    \end{itemize}
\end{itemize}
\end{document}
