\documentclass{article}
\usepackage[inline]{asymptote}
\usepackage{enumitem,amsmath,amsfonts,geometry,parskip,amssymb}
\def\asydir{asy}
\title{Induction and Divisibility}
\author{Andres Buritica Monroy}
\date{}
\begin{document}
\maketitle
\section{Induction and variants}
Arguably, the defining property of the integers is the \textbf{Principle of
	Mathematical Induction}: if we have a set $S\subseteq\mathbb N$ such that
$\forall a\in S,\ a+1\in S$ then $S=\mathbb N$.

To prove that some sentence $P(n)$ is true for all positive integers $n$, we
follow the following structure.
\begin{itemize}
	\item Prove that $P(1)$ is true.
	\item Prove that if $P(n)$ is true, then $P(n+1)$ is true.
\end{itemize}
If we've done both of those things, why can we conclude that $P(a)$ is true
for all $a\in\mathbb N$?

We may use induction to prove some foundational results about the integers:
\begin{itemize}
	\item For all positive integers $n$, $n\ge 1$.
	\item If $S$ is a nonempty set of positive integers, there is some $a\in
		      S$ such that for any $b\in S$ we have $a\le b$.
	      This is known as the \textbf{Well-Ordering Principle}.
\end{itemize}

The final addition to our induction toolkit will be \textbf{strong induction}:
if $S$ is a set of positive integers such that $1\in S$ and
\[\forall a\in \mathbb N,\ (\forall b\in\mathbb N,\ b\le a\implies b\in
	S)\implies a+1\in S,\]
then $\mathbb N=S$.
\section{Divisibility}
For integers $a$ and $b$, we say $a\mid b$ (read ``$a$ divides
$b$'') if there is some integer $c$ with $b=a\times c$.
\begin{itemize}
	\item Prove that if $a$ and $b$ are positive integers with $a\mid b$ then
	      $a\le b$.
	\item Prove that if $a\mid b$ and $a\mid c$ then $a\mid bx+cy$ for all
	      integers $x$ and $y$.
\end{itemize}

We define a \emph{prime} as a positive integer larger than 1 which is not
divisible by any positive integer other than 1 and itself.
\begin{itemize}
	\item Prove that every positive integer larger than 1 can be written as a
	      product of primes.
	\item Prove that there are infinitely many primes.
\end{itemize}
Also look for factorisations when trying problems: e.g.
\begin{align*}
	axy+bx+cy=d & \iff (ax+c)(ay+b)=ad+bc                                     \\
	a^k-b^k     & =(a-b)\left(a^{k-1}+a^{k-2}b+\cdots+b^{k-1}\right)          \\
	a^k+b^k     & =a^k-(-b)^k=(a-(-b))(a^{k-1}+a^{k-2}(-b)+\cdots+(-b)^{k-1}) \\
	            & =(a+b)(a^{k-1}-a^{k-2}b+\cdots+b^{k-1}),
\end{align*}
where the last factorisation is true when $k$ is odd.
\section{Problems}
\begin{enumerate}
	\item Given positive integers $n>1$ and $k$, prove that there are unique
	      nonnegative integers $m,a_0,a_1,\ldots,a_m$ such that $a_m>0,\ 0\le
		      a_i<n$ for all $i$, and
	      \[k=a_0 n^0+a_1 n^1+\cdots+a_m n^m.\]
	\item Find all right-angled triangles with positive integer sides
	      such that their area and perimeter are equal.
	\item Prove that $1^k+2^k+\cdots+n^k$ is divisible by $1+2+\cdots+n$ for
	      all positive integers $n$ and odd positive integers $k$.
	\item Prove that if $2^n+1$ is prime for a positive integer $n$, then $n$ is
	      a power of 2.
	\item Prove that if $a$ is an integer and $b$ is a positive integer, there
	      is a unique pair $(q,r)$ of integers such that $0\le r<b$ and $a=qb+r$.
	\item Let $m$ and $n$ be positive integers. Prove that $\sqrt[m]n$ is an
	      integer or irrational.
	\item Let $a,b,c$ be positive integers such that $a^3+b^3=2^c$. Prove that
	      $a=b$.
	\item Let $a$ and $b$ be positive integers such that $a\mid b^2\mid a^3\mid
		      b^4\mid\cdots$

	      Prove that $a=b$.
	\item Prove that every positive integer is a sum of one or more numbers of
	      the form $2^r3^s$, where $r$ and $s$ are nonnegative integers and no
	      summand divides another.
\end{enumerate}
\newpage
\section{Homework}
\begin{enumerate}
	\item Prove that for any natural number $n$, the fraction
	      \[\frac{21n+4}{14n+3}\]
	      is irreducible.
	\item Determine all pairs $(a,b)$ of non-negative integers such that
	      \[\frac{a+b}2-\sqrt{ab}=1.\]
	\item Prove that if there are two terms of an arithmetic progression which are
	      coprime integers, then there is an infinite subset of that
	      progression such that any two elements of that subset are coprime.
\end{enumerate}
\end{document}
