\documentclass{article}
\usepackage[inline]{asymptote}
\usepackage{enumitem,amsmath,amsfonts,geometry,parskip,amssymb}
\def\asydir{asy}
\title{Constructions and Existence Proofs}
\author{Andres Buritica Monroy}
\date{}
\begin{document}
\maketitle
\section{Existence Results}
Number theory has quite a few results about the existence of a number satisfying
certain properties that you should be aware of. We've already seen many of
these:
\begin{itemize}
  \item There are infinitely many primes.
  \item (Bezout's Identity)
    For any coprime integers $a$ and $b$, there are integers $x$ and $y$ such
    that $ax+by=1$.
  \item (Chinese Remainder Theorem)
    For any positive integers $a_1,\ldots,a_n$, and any pairwise coprime positive
    integers $b_1,\ldots,b_n$, there is exactly one residue $x\pmod {b_1\ldots
    b_n}$ such that $x\equiv a_i\pmod{b_i}$ for each $i$.
\end{itemize}
The pigeonhole principle isn't strictly a number-theoretic result but is also
often useful. 
\section{Techniques}
You will learn much of this by trying problems, but in brief:
\begin{itemize}
  \item Try small cases. TRY SMALL CASES!\@
  \item Build larger constructions from smaller ones; build constructions that
    satisfy more conditions from constructions that satisfy fewer.
  \item Try to control properties of your construction --- rely on simple
    properties as opposed to ``randomness'' (exception: pigeonhole).
\end{itemize}
\newpage
\section{Problems}
\begin{enumerate}
  \item Prove that for each positive integer $n$, there are $n$ consecutive
    positive integers, none of which is a prime power.
  \item
    \begin{enumerate}
      \item 
        Prove that for every positive integer $n$ there is a number divisible by
        $n$ consisting of only $1$s and $0$s.
      \item Prove that if $n$ is not a multiple of $5$, there is a number
        divisible by $n$ consisting of only $1$s and $2$s.
    \end{enumerate}
  \item Prove that for every positive integer $n$, there is a set $S$ of $n$
    distinct positive integers such that every subset of $S$ has a geometric
    mean which is a positive integer.
  \item Prove that there is an infinite set of positive integers such that the
    sum of any finite subset is not a perfect power.
  \item Prove that for every positive integer $n$, there are infinitely many
    terms of the Fibonacci sequence which are divisible by $n$.
  \item Prove that for every positive integer $n$, there is a positive integer
    $X$ such that \[X,\ 2X,\ 3X,\ \ldots,\ nX\] are all nontrivial perfect powers.
  \item Does there exist an infinite sequence of integers $a_1,a_2,\ldots$ such
    that $\gcd(a_m,a_n)=1\iff |m-n|=1$?
  \item Let $m$ and $c$ be integers.
    Prove that for any infinite sequence $a_1,a_2,\ldots$ of positive integers which
    contains every positive integer exactly once, there
    are integers $x,y,k$ such that $x<y$ and $a_x+a_{x+1}+\cdots+a_y=mk+c$.
\end{enumerate}
\newpage
\section{Homework}
\begin{enumerate}
  \item Prove that for any positive integer $c$ and any prime $p$, there is a
    positive integer $x$ such that $x^x\equiv c\pmod p$.
  \item Let $x$ be an irrational number, and let $a$ and $b$ be real numbers
    such that $0\le a<b\le 1$. Prove that there is an integer $n$ such that
    $a<\{nx\}<b$. Hence prove that there is a power of $2$ whose decimal
    representation starts with $2023$.
  \item Prove that every arithmetic progression $a,a+b,\ldots$ where
    $\gcd(a,b)=1$ has infinitely many terms which are not divisible by any
    perfect square larger than $1$.
\end{enumerate}
\end{document}
