\documentclass{article}
\usepackage[inline]{asymptote}
\usepackage{enumitem,amsmath,amsfonts,geometry,parskip,amssymb}
\def\asydir{asy}
\newcommand\Nn{\mathbb{N}}
\newcommand\Zz{\mathbb{Z}}
\title{Sequences and Integer Functions}
\author{Andres Buritica Monroy}
\date{}
\begin{document}
\maketitle
\section{Problems}
\begin{enumerate}
	\item Prove that there does not exist a function $f:\Nn\to\Nn$
	      such that for any distinct positive integers $i$ and $j$,
	      $\gcd(f(i)+j,f(j)+i)=1$.
	\item Consider a sequence of positive integers $a_1,a_2,\ldots$ which satisfies
	      $a_n=a_{n-1}^2+a_{n-2}^2+a_{n-3}^2$ for all $n\ge 3$. Prove that if
	      $a_k=1997$ then $k\le 4$.
	\item Prove that any infinite sequence of integers in arithmetic progression has an
	      infinite subsequence in geometric progression.
	\item Prove that there exists a strictly increasing sequence $\{a_n\}_1^\infty$
	      of positive integers such that for any $k\ge 0$, the sequence $\{k+a_n\}$
	      contains only finitely many primes.
	\item The sequence $\{a_i\}_1^\infty$ is defined by $a_1 = 1$ and
	      $a_{n+1}=a_n^2+1$ for $n\ge 1$. Prove that there are infinitely many primes
	      which divide some $a_i$.
	\item Let $n$ be a positive integer. Define a sequence by letting $a_1=n$, and
	      for each $i>1$ choosing $a_i$ such that $0\le a_i<i$ and
	      $\frac{a_1+\cdots+a_i}i$ is an integer. Prove that this sequence is
	      eventually constant.
\end{enumerate}
\newpage
\section{Homework}
\begin{enumerate}
	\item Find all functions $f:\Nn\to\Nn$ satisfying $f(n+f(n))=f(n)$ for all
	      $n$ such that 1 is in the range of $f$.
	\item Find all monotonically increasing functions $f:\Nn\to\Zz_{\ge 0}$ such that $f(mn)=f(m)+f(n)$ for
	      all nonnegative integers $m$ and $n$.
	\item Let $a,b$ be odd positive integers. Define the sequence $c_n$ by
	      choosing $c_1=a,c_2=b$ and for each $i>2$ letting $c_i$ be the largest odd
	      divisor of $c_{i-1}+c_{i-2}$. Prove that this sequence is eventually
	      constant
	      (that is, there is an $m$ such that for any $i,j>m$, $a_i=a_j$).
\end{enumerate}
\end{document}
