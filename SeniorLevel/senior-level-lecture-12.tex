\documentclass{article}
\usepackage[inline]{asymptote}
\usepackage{enumitem,amsmath,amsfonts,geometry,parskip,amssymb}
\def\asydir{asy}
\title{Integer Polynomials}
\newcommand\Nn{\mathbb{N}}
\newcommand\Zz{\mathbb{Z}}
\author{Andres Buritica Monroy}
\date{}
\begin{document}
\maketitle
\section{Basic facts}
We say that $p(x)$ is \emph{divisible} by a polynomial $q(x)$ if there is a
polynomial $r(x)$ (not necessarily with integer coefficients) such that
$p(x)=q(x)r(x)$. We also write $q(x)\mid p(x)$.

Examples:
\begin{itemize}
	\item If $p(x)=2x+2$, does $p(x)\mid x^2-1$?

	      One thing that this shows is that even if $p(x)\mid q(x)$ as
	      polynomials, substituting specific integers for $x$ won't necessarily
	      divisible numbers. For example, $p(2)=6$, which does not divide
	      $2^2-1=3$.
	\item If $p(x)=3x+1$, does $p(x)\mid 9x^2+1$?
\end{itemize}
For the following, $p$ is a polynomial with integer coefficients.
\begin{itemize}
	\item (Division algorithm) For any polynomial $q$ there are unique
	      polynomials $f$ and $r$ such that $p(x)=q(x)f(x)+r(x)$ and $\deg r<\deg q$.

	      If $q$ has integer coefficients, then $f$ and $r$ have rational
	      coefficients. If $q$ has a leading coefficient of $\pm 1$, then $f$ and $r$
	      have integer coefficients.

	      Examples:
	      \begin{itemize}
		      \item Let $p(x)=2x^3+x+1$, and $q(x)=x^2+3x-5$.
		      \item Let $p(x)=x^3+x+1$, and $q(x)=2x^2+3x-5$.
	      \end{itemize}
	\item (Factor theorem) If $a$ is a root of $p$, then $x-a\mid p(x)$.

	      ``$a$ is a root of $p$'' just means $p(a)=0$.
	\item (Remainder theorem) If $p(a)=c$ then $x-a\mid p(x)-c$.
	\item (*) If $a$ and $b$ are integers, then $a-b\mid p(a)-p(b)$. (This only
	      works if $p$ is an integer polynomial!)
	\item (Finite differences) If $c$ is a constant and $p$ is a polynomial with
	      leading coefficient $ax^n$, then $p(x)-p(x-c)$ is a polynomial with
	      leading coefficient $nacx^{n-1}$.
	\item (Rational Root Theorem) If $y$ and $z$ are integers with $\gcd(y,z)=1$
	      such that $p(y/z)=0$, and if $a_0$ is the constant term and $a_n$ is
	      the leading coefficient of $p$, then $y\mid a_0$ and $z\mid a_n$.
\end{itemize}
\section{Problems}
\begin{enumerate}
	\item Prove that every nonconstant integer polynomial has a composite number
	      in its image.
	\item Do there exist two quadratics $ax^2+bx+c$ and $(a+1)x^2+(b+1)x+(c+1)$
	      which both have two integer roots?
	\item Let $P$ be an integer polynomial such that if $P(x)$ is an integer then
	      $x$ is rational. Prove that $P$ is linear.
	\item Find all polynomials $P(x)$ with integer coefficients such that if
	      $m\mid n$ then $P(m)\mid P(n)$.
	\item Prove that for any two distinct polynomials $P$ and $Q$ with coefficients in
	      $\{0,1,\ldots,9\}$, either $P(-2)\ne Q(-2)$ or $P(-5)\ne Q(-5)$.
	\item Let $P(x)$ and $Q(x)$ be polynomials with integer coefficients such that the
	      leading coefficient of $P(x)$ is 1. Suppose that $P(n)^n$ divides $Q(n)^{n+1}$ for infinitely
	      many positive integers $n$.
	      Prove that $P(n)$ divides $Q(n)$ for infinitely many positive integers $n$.
	\item Let $n\ge 2$ be an integer and
	      $P(x)$ be a polynomial with nonnegative integer coefficients satisfying
	      $P(1)=1$ and $x^n P(1/x)=P(x)$ for all $x$.
	      Prove that there exist infinitely many
	      pairs $x, y$ of positive integers such that $x|P(y)$ and $y|P(x)$.
	\item  Given are positive integers $a, b$ satisfying $a \geq 2b$. Does there
	      exist a polynomial $P(x)$ of degree at least $1$ with coefficients from the
	      set $\{0, 1, 2, \ldots, b-1 \}$ such that $P(b) \mid P(a)$?
\end{enumerate}
\newpage
\section{Homework}
\begin{enumerate}
	\item Prove that if $P$ is a polynomial with integer coefficients and leading term $a_0n^k$ such
	      that $m\mid P(n)$ for all $n$, then $m\mid k!a_0$.
	\item Prove that for every polynomial $P(x)$ of degree at least 2 with integer
	      coefficients, there is an infinite arithmetic
	      progression of integers which does not contain $P(k)$ for any integer $k$.
	\item Is there a polynomial $f$ of degree $2023$ with integer
	      coefficients such that \[f(n), f(f(n)), f(f(f(n))), \cdots\] are pairwise
	      relatively prime for any integer $n$?
\end{enumerate}
\end{document}
