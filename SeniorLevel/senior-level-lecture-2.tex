\documentclass{article}
\usepackage[inline]{asymptote}
\usepackage{enumitem,amsmath,amsfonts,geometry,parskip,amssymb}
\def\asydir{asy}
\title{Division Algorithm, Euclid and Bezout}
\author{Andres Buritica}
\date{April 16, 2022}
\begin{document}
\maketitle
\section{Division Algorithm}
  From last time: 
  \begin{itemize}
    \item Prove that if $a$ is an integer and $b$ is a positive integer, there
      is a unique pair $(q,r)$ of integers such that $0\le r<b$ and $a=qb+r$.
  \end{itemize}
  A similar result is true for polynomials: if $A(x)$ and $B(x)$ are
  polynomials with $B\ne 0$, then there is a unique pair of polynomials $Q(x)$
  and $R(x)$ such that $\deg R<\deg B$ and \[A(x)=Q(x)B(x)+R(x).\]
  In particular, for sufficiently large $n$, $B(n)>R(n)$.

  We can find these polynomials by \emph{polynomial long division}.

  \begin{itemize}
    \item Prove that if there are polynomials $A,B,Q,R$ with integer
      coefficients satisfying
      \[A(x)=Q(x)B(x)+R(x),\] then for each $n$ we have \[B(n)\mid A(n)\iff
        B(n)\mid R(n).\]
    \item Find all integers $n$ such that $n^2+1\mid n^3+n^2-n-15$.
  \end{itemize}
\section{Euclid's Algorithm}
  We define the \emph{greatest common divisor} of two integers $a$ and $b$, not
  both of which are 0, as the largest positive integer $d$ such that $d\mid a$
  and $d\mid b$. We notate it by $\gcd(a,b)$.
  \begin{itemize}
    \item Let $a$ and $b$ be integers. Prove that if $a=qb+r$ then
      $\gcd(a,b)=\gcd(b,r)$.
    \item Find $\gcd(72,30)$ by applying the previous result repeatedly. (This
      is Euclid's Algorithm.)
    \item Prove that if we have a function $F:\mathbb N^2\to\mathbb R$ such that
      $F(a,a)=a\ \forall a,\ F(a,b)=F(b,a)\ \forall a,b$, and
      $F(a,b)=F(a,b-a)\ \forall a,b$ s.t. $b<a$
      then $F(a,b)=\gcd(a,b)\ \forall a,b$.
    \item Prove that for all positive integers $a,m,n$ we have
      \[\gcd(a^m-1,a^n-1)=a^{\gcd(m,n)}-1.\]
  \end{itemize}
\section{Bezout's Identity}
  \begin{itemize}
    \item Let $a$ and $b$ be integers. Prove that there are integers $c$ and $d$
      such that $ac+bd=\gcd(a,b)$.
    \item Prove that if $n,a,b$ are positive integers such that $n\mid ab$ and
      $\gcd(n,a)=1$ then $n\mid b$.
    \item Let $a$ and $b$ be integers with $\gcd(a,b)=1$, and let $c$ be an
      integer. Prove that if $a\mid c$ and $b\mid c$ then $ab\mid c$.
    \item Let $a,b,c,d$ be positive integers with $\gcd(a,b)=1$. Prove that there is an integer $e$
      such that $a\mid e-c$ and $b\mid f-d$.
  \end{itemize}
\section{Problems}
  \begin{enumerate}
    \item Let $a,b,c,d$ be positive integers with $ab=cd$. Prove that there
      exist positive integers $p,q,r,s$ such that \(a=pq,b=rs,c=pr,d=qs\).
    \item Prove that for positive integers $m,n>2$ we cannot have \[2^m-1\mid
        2^n+1.\]
    \item Find all positive integers $n$ such that $3^{n-1}+5^{n-1}\mid
      3^n+5^n$.
    \item Let $n$ be a composite positive integer. Calculate
      \[\gcd((n-1)!+1,n!).\]
    \item Assume that $p_1,\ldots,p_m,\ q_1,\ldots,q_n$ are primes such that
      \[p_1p_2\cdots p_m=q_1q_2\cdots q_n.\]
      Prove that the $q_i$s are a permutation of the $p_i$s.
    \item Find all pairs of positive integers $a,b$ such that \[b^2-a\mid
        a^2+b\quad\text{and}\quad a^2-b\mid b^2+a.\]
    \item Let $m$ and $n$ be positive integers. Prove that
      \[m\mid\gcd(m,n)\binom mn.\]
    \item Find all pairs of positive integers $x,y$ such that $xy^2+y+7\mid
      x^2y+x+y$.
  \end{enumerate}
\newpage
\section{Homework}
  \begin{enumerate}
    \item
      \begin{enumerate}
        \item Find all integers $x,y$ such that $6x+2y=8$.
        \item Find all integers $x,y$ such that $6x+4y=8$.
      \end{enumerate}
    \item Let $a_1,a_2,\ldots,a_n$ be positive integers, and let $d$ be the
      largest positive integer such that $d\mid a_i$ for all $i$.

      Prove that there are integers $b_1,b_2,\ldots,b_n$ such that
      \[d=a_1b_1+a_2b_2+\cdots+a_n b_n.\]
    \item The Fibonacci sequence is defined by $F_1=F_2=1$ and
      \[F_{n+1}=F_n+F_{n-1}.\]
      Prove that $\gcd(F_n,F_{n+2})=1$ for all $n$.
  \end{enumerate}
\end{document}
