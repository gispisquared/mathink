\documentclass{article}
\usepackage[inline]{asymptote}
\usepackage{enumitem,amsmath,amsfonts,amssymb,geometry,parskip}
\def\asydir{asy}
\newcommand\lcm{\text{lcm}}
\title{Prime factorisations and Arithmetic Functions}
\author{Andres Buritica}
\date{}
\begin{document}
\maketitle
\section{Prime factorisations}
  From last time:
  \begin{itemize}
    \item Assume that $p_1,\ldots,p_m,\ q_1,\ldots,q_n$ are primes such that
      \[p_1p_2\cdots p_m=q_1q_2\cdots q_n.\]
      Prove that the $q_i$s are a permutation of the $p_i$s.
  \end{itemize}
  Therefore, each positive integer has a unique prime factorisation (the
  Fundamental Theorem of Arithmetic).
  In particular we can write a positive integer $n$ uniquely as
  \[n=p_1^{e_1}p_2^{e_2}\cdots p_k^{e_k},\]
  where $p_i$ are all prime and $e_i$ are all positive integers.

  Prime factorisations allow us to view statements about divisibility and
  multiplication in terms of the exponents $e_i$.

  In what follows, let 
  \[a=p_1^{e_1}p_2^{e_2}\cdots p_m^{e_m},\
        b=q_1^{f_1}q_2^{f_2}\cdots q_k^{e_k}.\]
  \begin{itemize}
    \item $a\mid b$ if and only if for each $i$ we have that $p_i=q_j$
      for some $j$, and that $e_i\le f_j$.
    \item $a$ is a perfect $k$th power if and only if $k\mid e_i$ for all $i$.
    \item The lcm is found by taking the maximum power of each prime that
      divides either $a$ or $b$; the gcd is found by taking the minimum power of
      each prime that divides both $a$ and $b$.
    \item $\gcd(a,b)\times\lcm(a,b)=ab$.
  \end{itemize}
  \newpage
\section{Arithmetic functions}
  We define:
  \begin{itemize}
    \item The number of positive divisors function $d(n)$.
    \item The sum of positive divisors function $\sigma(n)$.
    \item The totient function $\varphi(n)$: the number of positive integers
      which are at most $n$ and coprime to $n$.
  \end{itemize}
  A function $f:\mathbb N\to\mathbb R$ is called multiplicative if for any
  coprime positive integers $a$ and $b$, we have
  \[f(a)f(b)=f(ab).\]
  It's called completely multiplicative if this equation holds for \emph{any}
  positive integers $a$ and $b$
  \begin{itemize}
    \item Prove that the values at the primes of a completely multiplicative
      function completely define the function (unless these values are all 0, in
      which case $f(1)$ can be 0 or 1).
    \item Prove that the values at prime powers of a multiplicative function
      completely define it (once again, unless these values are all 0).
    \item Prove that $d$ and $\sigma$ are multiplicative. ($\varphi$ is also
      multiplicative, but we will prove this next term.)
    \item
      Find formulae for $d(n),\sigma(n),\varphi(n)$ where
      $n=p_1^{e_1}p_2^{e_2}\cdots p_k^{e_k}$.
  \end{itemize}
\section{Problems}
\begin{enumerate}
  \item Prove that $d(n)\le 2\sqrt n$.
  \item Let $a,b,p$ be positive integers such that $p$ is prime and
    $\lcm(a,a+p)=\lcm(b,b+p)$. Prove that $a=b$.
  \item Prove that for all $n$,
    \[\sigma(1)+\sigma(2)+\cdots+\sigma(n)\le n^2.\]
  \item Prove that if $ab$ is a perfect square, then so are $\frac a{\gcd(a,b)}$ and
      $\frac b{\gcd(a,b)}$.
  \item Prove that for all composite $n$ apart from 6,
    \[\sqrt n\le\varphi(n)\le n-\sqrt n.\]
  \item Let $n$ be an even positive integer such that $\sigma(n)=2n$. Prove that
    $n=2^{p-1}\left(2^p-1\right)$, where $p$ is a prime.
  \item For any positive integer $n$, prove that $\displaystyle\sum_{d\mid
    n}\varphi(d)=n$.
\end{enumerate}
\newpage
\section{Homework}
\begin{enumerate}
  \item The cells in a jail are numbered from 1 to 100, and there are 100
    buttons also numbered from 1 to 100. For each $i$, the $i$th button opens a
    closed cell and closes an open cell, affecting only the multiples of $i$. 

    For example, if the 47th cell is open and the 94th cell is closed, then
    pressing the 47th button will close the 47th cell and open the 94th cell.

    Initially all cells are closed.
    The warden presses the first button, then the second, and so on, for all
    100 buttons. Which cells are open at the end?
  \item Find all primes $p$ such that $p^{2022}+p^{2023}$ is a perfect square.
  \item Prove that for any positive integer $n$ we have $\sigma(n)\ge d(n)\sqrt
    n$.
\end{enumerate}
\end{document}
