\documentclass{article}
\usepackage[inline]{asymptote}
\usepackage{enumitem,amsmath,amsfonts,amssymb,geometry,parskip}
\def\asydir{asy}
\title{Modular Arithmetic 1}
\author{Andres Buritica Monroy}
\date{}
\begin{document}
\maketitle
\section{Residue Classes}
  Let $n$ be a nonzero integer. For integers $a$ and $b$, we say that
  \[a\equiv b\pmod n\iff n\mid b-a.\]
  Notice that for fixed values of $a$ and $n$, infinitely many values of $b$
  satisfy $a\equiv b\pmod n$.

  The numbers $0,1,\ldots,n-1$ are called the \emph{least residues mod $n$}. 
  Every integer is congruent to a unique least residue mod $n$.
  \begin{itemize}
    \item Find the least residue of $81$ mod $7$.
    \item Find the least residue of $-1$ mod $2023$.
  \end{itemize}
\section{Operations}
  Prove that if $a\equiv b\pmod n$ and $c\equiv d\pmod n$ then
  \begin{itemize}
    \item $a+c\equiv b+d\pmod n$
    \item $a-c\equiv b-d\pmod n$
    \item $ac\equiv bd\pmod n$
    \item $a^m\equiv b^m\pmod n$ for any nonnegative integer $m$.
  \end{itemize}
  Let $a,n,x,y$ be integers such that $a\mid x,y$.
  \begin{itemize}
    \item Find a counterexample to the statement that if $x\equiv y\pmod n$ then $x/a\equiv
      y/a\pmod n$.
    \item Find some $m$ in terms of $a$ and $n$ such that you can guarantee that
      $x/a\equiv y/a\pmod m$.
  \end{itemize}
\section{Multiplication by coprime residues}
  Let $a$ and $n$ be positive integers such that $\gcd(a,n)=1$.
  \begin{itemize}
    \item Prove that if $\gcd(y,n)=1$ then there exists some $x$ such that
      $ax\equiv y\pmod n$.
    \item Prove that $a^{\varphi(n)}\equiv 1\pmod n$. 
  \end{itemize}
  This last dot point is known as Euler's
    Theorem. If $n=p$ is prime, then $\varphi(p)=p-1$ so $a^{p-1}\equiv 1\pmod
    p$, which is known as Fermat's Little Theorem.
\section{Problems}
\begin{enumerate}
  \item Find all primes $p$ such that $29^p+1$ is a multiple of $p$.
  \item Let $n>6$ be an integer such that $n-1$ and $n+1$ are both prime. Prove
    that $720\mid n^2(n^2+16)$.
  \item Let $a_1=20,\ a_2=23$. For $n\ge 1$, let $a_{n+1}$ be the least residue
    of $a_n+a_{n-1}$ mod $100$. Find the least residue of
    $a_1^2+\cdots+a_{2023}^2\bmod 8$.
  \item Let $n$ be a positive integer. All numbers $m$ which are coprime to $n$
    satisfy $m^2\equiv 1\pmod n$. Find the maximum possible value of $n$.
  \item Find all primes $p$ such that $2^{p-2}+1$ is a multiple of $p$.
  \item Show that $30$ is the greatest common divisor of all numbers of the form
    $2^{3n}+5^{n+1}+3^{n+2}$, where $n \in \mathbb{N}$.
  \item Define the sequence $a_n=2^n+3^n+6^n-1,\ n\in\mathbb N$.
    Find all primes which do not divide $a_n$ for any $n$.
\end{enumerate}
\newpage
\section{Homework}
\begin{enumerate}
  \item Let $S$ be a subset of the set of numbers $\{1, 2, 3,\ldots, 2023\}$
    such that if $a,b$ are in $S$, then $23\nmid a+b$. What is the maximum
    possible size of $S$?
  \item Prove that every positive integer has at least as many divisors which
    are 1 (mod 4) as divisors which are 3 (mod 4).
  \item Does one of the first $10^8+1$ Fibonacci numbers end with four zeroes?
\end{enumerate}
\end{document}
