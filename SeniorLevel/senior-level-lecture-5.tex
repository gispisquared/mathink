\documentclass{article}
\usepackage[inline]{asymptote}
\usepackage{enumitem,amsmath,amsfonts,amssymb,geometry,parskip}
\def\asydir{asy}
\title{Modular Arithmetic 2}
\author{Andres Buritica Monroy}
\date{}
\begin{document}
\maketitle
\section{Inverses}
Let $a$ and $n$ be positive integers such that $\gcd(a,n)=1$.
\begin{itemize}
	\item Prove that there exists a unique least residue $a^{-1}\pmod n$ such that
	      $aa^{-1}\equiv 1\pmod n$.
	\item Use this fact to give another proof that if $ax\equiv ay\pmod n$, then
	      $x\equiv y\pmod n$.
\end{itemize}
The least residue $a^{-1}$ such that $aa^{-1}\equiv 1\pmod n$ is called the \emph{inverse} of
$a$ mod $n$.

Let $a,b,c,d,k,n$ be positive integers such that
$\gcd(b,n)=\gcd(d,n)=1$. Prove that
\begin{itemize}
	\item $b^{-1}d^{-1}\equiv (bd)^{-1}\pmod n$
	\item $(b^k)^{-1}\equiv (b^{-1})^k\pmod n$
	\item $ab^{-1}+cd^{-1}\equiv (ad+bc)(bd)^{-1}\pmod n$
\end{itemize}
\section{More Theorems}
\begin{itemize}
	\item (Wilson's Theorem) Prove that $(n-1)!\equiv -1\pmod n$ if and only if $n$ is prime.
	\item (GCD Trick) Prove that if $a^x\equiv 1\pmod n$ and $a^y\equiv 1\pmod n$ then
	      $a^{\gcd(x,y)}\equiv 1\pmod n$.
	\item (Chinese Remainder Theorem) Let $a_1,a_2,\ldots,a_k$ be pairwise coprime positive integers, and let
	      $b_1,b_2,\ldots,b_k$ be integers. Prove that there is exactly one least
	      residue $x$ mod $a_1a_2\cdots a_n$ such that for each $i$,
	      \[b_i\equiv x\pmod {a_i}.\]
	\item (Euler's product formula) Prove that if $\gcd(a,b)=1$ then $\varphi(ab)=\varphi(a)\varphi(b)$. Use
	      this fact to find a formula for $\varphi(n)$ in terms of the prime
	      factorisation of $n$.
\end{itemize}
\section{Problems}
\begin{enumerate}
	\item Let $p=3k-1$ be a prime. Prove that
	      \[1^{-1}-2^{-1}+3^{-1}-4^{-1}+\cdots+(2k-1)^{-1}\equiv 0\pmod p.\]
	\item Prove that for each positive integer $n$ there exist $n$ consecutive
	      positive integers, none of which is a prime power.
	\item Call a lattice point ``visible'' if the greatest
	      common divisor of its coordinates is 1. Prove that there exists a 100 × 100
	      square on the board none of whose points are visible.
	\item We are given a positive integer $s \ge 2$. For each positive integer
	      $k$, we define its twist $k'$ as follows: write $k$ as $as+b$, where $a, b$
	      are non-negative integers and $b < s$, then $k' = bs+a$. For the positive
	      integer $n$, consider the infinite sequence $d_1, d_2, \dots$ where $d_1=n$
	      and $d_{i+1}$ is the twist of $d_i$ for each positive integer $i$.
	      Prove that this sequence contains $1$ if and only if the remainder when $n$
	      is divided by $s^2-1$ is either $1$ or $s$.
	\item Let $p>3$ be prime. Define $m=(4^p-1)/3$. Prove that $2^{m-1}\equiv
		      1\pmod m$.
	\item Define a sequence by $a_1=n$ and $a_{i+1}=\frac{a_i(a_i-1)}2$ for each
	      $i\ge 1$. For which positive integers $n$ are all values of $a_i$ odd?
\end{enumerate}
\newpage
\section{Homework}
\begin{enumerate}
	\item Compute the remainder when $2023^{2022}$ is divided by $2021$.
	\item
	      We define
	      \[\binom nk=\frac{n!}{k!(n-k)!}.\]

	      For each prime $p$ and positive integer $k$, find the least residues of
	      \[\binom {p-1}k\quad\text{and}\quad\frac1p\binom pk\] in mod $p$.
	\item Prove that if $p$ is an odd prime that divides $n^2+1$ for some integer
	      $n$, then $p\equiv 1\pmod 4$.
\end{enumerate}
\end{document}
