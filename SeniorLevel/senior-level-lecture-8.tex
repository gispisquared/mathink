\documentclass{article}
\usepackage[inline]{asymptote}
\usepackage{enumitem,amsmath,amsfonts,geometry,parskip,amssymb}
\def\asydir{asy}
\title{Descent and Floor Functions}
\author{Andres Buritica}
\date{}
\begin{document}
\maketitle
\section{Techniques}
  Descent works like this: assign some positive integer quantity
  (the ``size'' of a solution, e.g.\ the sum of the absolute values of the
  variables) to each solution, take the solution with lowest
  ``size'' that doesn't fit your claimed pattern and derive a contradiction.

  Floor functions: recall $\lfloor x\rfloor$ is the largest integer which is at
  most $x$. Most often, problems which involve $\lfloor x\rfloor$ will be
  solved by considering $x-\lfloor x\rfloor$ (also denoted $\{x\}$), or $\lceil
  x\rceil-x$, and doing inequalities.
\section{Problems}
\begin{enumerate}
  \item Find all solutions in integers to $a^3+2b^3+4c^3=0$.
  \item Prove that $\lfloor\sqrt n+\sqrt{n+1}\rfloor=\lfloor\sqrt{4n+1}\rfloor$.
  \item Prove that if $x, y, z$ are integers such that $x^2+y^2+z^2=(xy)^2$,
      then $x=y=0$.
  \item Let $a$ and $b$ be irrational numbers such that $\frac 1a+\frac 1b=1$.
    Let $A=\{\lfloor na\rfloor: n\in\mathbb N\}$, and $B=\{\lfloor nb\rfloor:
    n\in\mathbb N\}$. Prove that the sets $A$ and $B$ together contain each
    integer exactly once.
  \item Find all positive integers $n$ such that $1+\lfloor\sqrt n\rfloor$
    divides $n$.
  \item Prove that for any positive integer $n$ which is not a perfect square,
    there is a positive integer $k$ such that
    \[n=\left\lfloor n+\sqrt n+\frac12\right\rfloor.\]
  \item Let $d$ be a positive integer which is not a perfect square. Prove that
      there exist positive integers $x_0,y_0$ such that for any pair $x,y$ of
      positive integers satisfying $x^2-dy^2=1$, there is a positive integer $k$
      satisfying
      \[x+y\sqrt d=(x_0+y_0\sqrt d)^k.\]
\end{enumerate}
\newpage
\section{Homework}
  \begin{enumerate}
    \item Let's say you have a set $S$ of positive rational numbers such that
      $1\in S$, and if $x\in S$ then both $x+1$ and $\frac 1x$ are in $S$. Prove
      that $S$ contains all positive rationals.
    \item Let $p$ and $q$ be coprime. Prove that
      \[\sum_{i=1}^{q-1}\left\lfloor\frac{ip}{q}\right\rfloor=\frac{(p-1)(q-1)}2.\]
    \item A list of $2022$ positive integers is given, such that if you remove
      any one of them, the rest can be split into two groups of equal sum.
      Prove that all the numbers in the list are equal.
  \end{enumerate}
\end{document}
