\documentclass{article}
\usepackage[inline]{asymptote}
\usepackage{enumitem,amsmath,amsfonts,geometry,parskip,amssymb}
\def\asydir{asy}
\title{Number Bases}
\author{Andres Buritica Monroy}
\date{}
\begin{document}
\maketitle
\section{Definitions}
Let $b>1$ be an integer. For any integer $n\ge 1$ there are unique nonnegative
integers $k,a_0,\ldots,a_k$ such that each $a_i$ is less than $b$,
$a_k\ne 0$, and \[n=\sum a_i b^i.\]

This is called the \emph{base-$b$ representation} of $n$, and we write
\[n=\overline{a_k a_{k-1}\ldots a_0}_{(b)}.\]
The usual decimal representation takes $b=10$.

The sum of the digits of $n$ in base $b$ is denoted $s_b(n)$.
\section{Divisibility tests}
Since powers of $b$ have nice properties in modular arithmetic, base-$b$ expansions
naturally lead to simple divisibility tests. The most familiar of these are in
base $10$:
\begin{itemize}
	\item A number is divisible by $2^n$ iff the number produced by taking its
	      last $n$ digits is divisible by $2^n$.
	\item The same statement, replacing $2^n$ by $5^n$.
	\item A number is divisible by $3$ iff the sum of its digits is divisible by
	      $3$.
	\item The same statement, replacing $3$ by $9$.
	\item A number is divisible by $11$ iff the alternating sum of its digits
	      (that is, the sum of $(-1)^i a_i$) is divisble by $11$.
\end{itemize}
Using the language of modular arithmetic, we can generalise these statements to
arbitrary bases.
\begin{itemize}
	\item Let $l<k$. Then,
	      $\overline{a_k a_{k-1}\ldots a_0}_{(b)}\equiv \overline{a_{l-1}\ldots
			      a_0}_{(b)}\pmod{b^l}$. In particular, this congruence is still true
	      $\pmod{d^l}$ for any $d\mid b$.
	\item $n\equiv s_b(n)\pmod{b-1}$.
	\item $n\equiv a_0-a_1+a_2-\cdots+(-1)^k a_k\pmod {b+1}$.
\end{itemize}
For primes $p$ which do not divide any of $b-1,b,b+1$,
we can create divisibility tests as follows. Let $m$ be
such that $p\mid bm-1$. Then $p\mid bx+y\iff p\mid bmx+my\iff p\mid
	x+my$. For example, $7\mid 10a+b\iff 7\mid a-2b$.
\section{Problems}
\begin{enumerate}
	\item Find all $b$ such that $\overline{111}_{(b)}=\overline{212}_{(b-2)}$.
	\item A faulty car odometer always skips the digit $4$. If the odometer
	      reads $2005$, how many kilometres has the car actually travelled?
	\item Prove that $s_b(x+y)\le s_b(x)+s_b(y)$ and $s_b(xy)\le s_b(x)s_b(y)$.
	\item Prove that $\overline{11\ldots1}_{(9)}$ is always a triangular number.
	\item Can a number consisting of $5$ distinct even digits be a perfect
	      square?
	\item Prove that for any base $b$, there is a perfect square which ends with
	      $b$ distinct digits when written in base $b$.
	\item
	      \begin{enumerate}
		      \item Prove that for every positive integer $n$ and any base $b$
		            there is a number divisible by $n$ consisting of only $1$s and $0$s in
		            base $b$.
		      \item Prove that if $b$ is even and $n$ is odd, there is a number divisible by $n$
		            consisting of only odd digits in base $b$.
	      \end{enumerate}
	\item A sequence $\{x_n\}$ begins with $x_0=0$, and for each $i\ge 1$,
	      $x_{i+1}$ is the smallest positive integer greater than $x_i$ such that
	      the set $\{x_0,\ldots,x_{i+1}\}$ does not contain any 3-term arithmetic
	      progressions. Find $x_{2000}$.
\end{enumerate}
\newpage
\section{Homework}
\begin{enumerate}
	\item Find all $b$ such that
	      $\overline{234}_{(b+1)}-\overline{234}_{(b-1)}=70$.
	\item Find $s_{10}(s_{10}(s_{10}(4444^{4444})))$.
	\item Find all $b$ such that $\overline{11111}_{(b)}$ is a perfect square.
\end{enumerate}
\end{document}
