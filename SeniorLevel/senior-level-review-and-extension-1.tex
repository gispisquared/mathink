\documentclass{article}
\usepackage[inline]{asymptote}
\usepackage{enumitem,amsmath,amsfonts,amssymb,geometry,parskip,hyperref}
\hypersetup{colorlinks=true, urlcolor=blue}
\newcommand\lcm{\mathrm{lcm}}
\def\asydir{asy}
\title{Review and Extension --- Divisibility}
\author{Andres Buritica Monroy}
\date{}
\begin{document}
\maketitle
\section{Key concepts for this term}
\begin{itemize}
	\item Induction, strong induction and well-ordering.
	\item If $a\mid b$ and $a\mid c$ then for any integers $x$ and $y$, $a\mid
		      bx+cy$.
	\item Factorisations:
	      \begin{align*}
		      axy+bx+cy=d\iff (ax+c)(ay+b)=ad+bc \\
		      a^k-b^k=(a-b)\left(a^{k-1}+a^{k-2}b+\cdots+b^{k-1}\right)
	      \end{align*}
	\item Division Algorithm: for any integer $a$ and positive integer $b$ there
	      exist unique integers $q$ and $r$ such that $a=qb+r$ and $0\le r<b$.
	\item Euclid's Algorithm: if $a=qb+r$ then $\gcd(a,b)=\gcd(b,r)$.
	\item Bezout's Identity: for any integers $a$ and $b$ there exist integers
	      $x$ and $y$ such that $\gcd(a,b)=ax+by$.
	\item Fundamental Theorem of Arithmetic: prime factorisations are unique.
	\item When dealing with divisibility problems it's often more convenient to
	      think of numbers in terms of prime factorisations.
\end{itemize}
\section{Arithmetic Functions}
We define:
\begin{itemize}
	\item The number of positive divisors function $d(n)$.
	\item The sum of positive divisors function $\sigma(n)$.
\end{itemize}
Find formulae for $d(n)$ and $\sigma(n)$ where
$n=p_1^{e_1}p_2^{e_2}\cdots p_k^{e_k}$.
\section{Homework}
Solve and submit any three problems from the Problems sections of this
term's handouts that weren't covered in class.

Also complete the feedback form:
\url{https://forms.gle/fR78jMzBeHKWy7nM7}
\end{document}
