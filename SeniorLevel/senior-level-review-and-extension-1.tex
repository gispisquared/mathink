% chktex-file 3
\documentclass{article}
\usepackage[inline]{asymptote}
\usepackage{enumitem,amsmath,amsfonts,amssymb,geometry,parskip}
\newcommand\lcm{\mathrm{lcm}}
\def\asydir{asy}
\title{Review and Extension --- Divisibility}
\author{Andres Buritica}
\date{June 19, 2022}
\begin{document}
\maketitle
\section{Exam review}
\newpage
\section{Key concepts for this term}
  \begin{itemize}
    \item Induction, strong induction and well-ordering.
    \item If $a\mid b$ and $a\mid c$ then for any integers $x$ and $y$, $a\mid
      bx+cy$.
    \item Factorisations:
        \begin{align*}
          axy+bx+cy=d\iff (ax+c)(ay+b)=ad+bc \\
          a^k-b^k=(a-b)\left(a^{k-1}+a^{k-2}b+\cdots+b^{k-1}\right)
        \end{align*}
    \item Division Algorithm: for any integer $a$ and positive integer $b$ there
      exist unique integers $q$ and $r$ such that $a=qb+r$ and $0\le r<b$.
    \item Euclid's Algorithm: if $a=qb+r$ then $\gcd(a,b)=\gcd(b,r)$.
    \item Bezout's Identity: for any integers $a$ and $b$ there exist integers
      $x$ and $y$ such that $\gcd(a,b)=ax+by$.
    \item Fundamental Theorem of Arithmetic: prime factorisations are unique.
    \item When dealing with divisibility problems it's often more convenient to
      think of numbers in terms of prime factorisations.
    \item The definitions of multiplicative and completely multiplicative
      functions, and how to find them if you're given the values at prime
      powers.
    \item $d$, $\sigma$ and $\varphi$ are multiplicative.
    \item Formulas for $d$, $\sigma$ and $\varphi$.
  \end{itemize}
\section{Problems}
\begin{enumerate}
  \item Recall that $d(n)$ is the number of positive divisors of $n$, $\sigma(n)$ is
    the sum of the positive divisors of $n$, and $\varphi(n)$ is the number of
    positive integers which are at most $n$ and coprime to $n$.

    For a prime $p$ and positive integer $k$, find $d(p^k)$, $\sigma(p^k)$ and
    $\varphi(p^k)$.
  \item Let $S$ be a nonempty set of integers such that if $a$ and $b$ are in $S$, then
    so is $2a-b$.

    Prove that $S$ is an arithmetic
    progression. (That is, there are integers $a$ and $d$ such that $n\in S$ if
    and only if 
    $n=a+kd$ for some integer $k$.)
  \item Find all right-angled triangles with positive integer sides 
      such that their area and perimeter are equal.
  \item Prove that if $ab$ is a perfect square, then so are $\frac a{\gcd(a,b)}$ and
      $\frac b{\gcd(a,b)}$.
  \item Let $a,b,c,d$ be positive integers with $ab=cd$. Prove that there
    exist positive integers $p,q,r,s$ such that \(a=pq,b=rs,c=pr,d=qs\).
\end{enumerate}
\newpage
\section{Homework}
Instructions: solve and submit any three of these.
\begin{enumerate}
  \item Prove that $1^k+2^k+\cdots+n^k$ is divisible by $1+2+\cdots+n$ for
    all positive integers $n$ and odd positive integers $k$.
  \item Prove that if $2^n+1$ is prime for a positive integer $n$, then $n$ is
    a power of 2.
  \item Let $a,b,c$ be positive integers such that $a^3+b^3=2^c$. Prove that
    $a=b$.
  \item Let $a$ and $b$ be positive integers such that $a\mid b^2\mid a^3\mid
    b^4\mid\cdots$

    Prove that $a=b$.
  \item Prove that every positive integer is a sum of one or more numbers of
    the form $2^r3^s$, where $r$ and $s$ are nonnegative integers and no
    summand divides another.
  \item Prove that for positive integers $m,n>2$ we cannot have \[2^m-1\mid
      2^n+1.\]
  \item Find all positive integers $n$ such that $3^{n-1}+5^{n-1}\mid
    3^n+5^n$.
  \item Find all pairs of positive integers $a,b$ such that \[b^2-a\mid
      a^2+b\quad\text{and}\quad a^2-b\mid b^2+a.\]
  \item Let $m$ and $n$ be positive integers. Prove that
    \[m\mid\gcd(m,n)\binom mn.\]
  \item Find all pairs of positive integers $x,y$ such that $xy^2+y+7\mid
    x^2y+x+y$.
  \item Prove that $d(n)\le 2\sqrt n$ for all $n$.
  \item Let $a,b,p$ be positive integers such that $p$ is prime and
    $\lcm(a,a+p)=\lcm(b,b+p)$. Prove that $a=b$.
  \item Prove that for all $n$,
    \[\sigma(1)+\sigma(2)+\cdots+\sigma(n)\le n^2.\]
  \item Prove that for all composite $n$ apart from 6,
    \[\sqrt n\le\varphi(n)\le n-\sqrt n.\]
  \item Let $n$ be an even positive integer such that $\sigma(n)=2n$. Prove that
    $n=2^{p-1}\left(2^p-1\right)$, where $p$ is a prime.
  \item For any positive integer $n$, prove that $\displaystyle\sum_{d\mid
    n}\varphi(d)=n$.
\end{enumerate}
\end{document}
