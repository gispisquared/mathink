\documentclass{article}
\usepackage[inline]{asymptote}
\usepackage{enumitem,amsmath,amsfonts,amssymb,geometry,parskip,hyperref}
\hypersetup{colorlinks=true, urlcolor=blue}
\newcommand\lcm{\mathrm{lcm}}
\def\asydir{asy}
\title{Review and Extension --- Divisibility}
\author{Andres Buritica}
\date{}
\begin{document}
\maketitle
\section{Key concepts for this term}
  \begin{itemize}
    \item Induction, strong induction and well-ordering.
    \item If $a\mid b$ and $a\mid c$ then for any integers $x$ and $y$, $a\mid
      bx+cy$.
    \item Factorisations:
        \begin{align*}
          axy+bx+cy=d\iff (ax+c)(ay+b)=ad+bc \\
          a^k-b^k=(a-b)\left(a^{k-1}+a^{k-2}b+\cdots+b^{k-1}\right)
        \end{align*}
    \item Division Algorithm: for any integer $a$ and positive integer $b$ there
      exist unique integers $q$ and $r$ such that $a=qb+r$ and $0\le r<b$.
    \item Euclid's Algorithm: if $a=qb+r$ then $\gcd(a,b)=\gcd(b,r)$.
    \item Bezout's Identity: for any integers $a$ and $b$ there exist integers
      $x$ and $y$ such that $\gcd(a,b)=ax+by$.
    \item Fundamental Theorem of Arithmetic: prime factorisations are unique.
    \item When dealing with divisibility problems it's often more convenient to
      think of numbers in terms of prime factorisations.
  \end{itemize}
\section{Arithmetic Functions}
  We define:
  \begin{itemize}
    \item The number of positive divisors function $d(n)$.
    \item The sum of positive divisors function $\sigma(n)$.
    \item The totient function $\varphi(n)$: the number of positive integers
      which are at most $n$ and coprime to $n$.
  \end{itemize}
  Find formulae for $d(n)$ and $\sigma(n)$ where
    $n=p_1^{e_1}p_2^{e_2}\cdots p_k^{e_k}$.
\section{Problems}
\begin{enumerate}
  \item Given positive integers $n>1$ and $k$, prove that there are unique
    nonnegative integers $m,a_0,a_1,\ldots,a_m$ such that $a_m>0,\ 0\le
    a_i<n$ for all $i$, and
    \[k=a_0 n^0+a_1 n^1+\cdots+a_m n^m.\]
  \item Prove that $1^k+2^k+\cdots+n^k$ is divisible by $1+2+\cdots+n$ for
    all positive integers $n$ and odd positive integers $k$.
  \item Prove that if $2^n+1$ is prime for a positive integer $n$, then $n$ is
    a power of 2.
  \item Let $m$ and $n$ be positive integers. Prove that $\sqrt[m]n$ is an
    integer or irrational.
  \item Let $a,b,c$ be positive integers such that $a^3+b^3=2^c$. Prove that
    $a=b$.
  \item Let $a$ and $b$ be positive integers such that $a\mid b^2\mid a^3\mid
    b^4\mid\cdots$

    Prove that $a=b$.
  \item Prove that every positive integer is a sum of one or more numbers of
    the form $2^r3^s$, where $r$ and $s$ are nonnegative integers and no
    summand divides another.
  \item Prove that for positive integers $m,n>2$ we cannot have \[2^m-1\mid
      2^n+1.\]
  \item Find all positive integers $n$ such that $3^{n-1}+5^{n-1}\mid
    3^n+5^n$.
  \item Let $n$ be a composite positive integer. Calculate
    \[\gcd((n-1)!+1,n!).\]
  \item Let $S$ be a nonempty set of integers such that if $a$ and $b$ are in $S$, then
    so is $2a-b$.

    Prove that $S$ is an arithmetic
    progression. (That is, there are integers $a$ and $d$ such that $n\in S$ if
    and only if 
    $n=a+kd$ for some integer $k$.)
  \item Find all pairs of positive integers $a,b$ such that \[b^2-a\mid
      a^2+b\quad\text{and}\quad a^2-b\mid b^2+a.\]
  \item Let $m$ and $n$ be positive integers. Prove that
    \[m\mid\gcd(m,n)\binom mn.\]
  \item Find all pairs of positive integers $x,y$ such that $xy^2+y+7\mid
    x^2y+x+y$.
  \item Prove that $d(n)\le 2\sqrt n$ for all $n$.
  \item Let $a,b,p$ be positive integers such that $p$ is prime and
    $\lcm(a,a+p)=\lcm(b,b+p)$. Prove that $a=b$.
  \item Prove that for all $n$,
    \[\sigma(1)+\sigma(2)+\cdots+\sigma(n)\le n^2.\]
  \item Prove that if $ab$ is a perfect square, then so are $\frac a{\gcd(a,b)}$ and
      $\frac b{\gcd(a,b)}$.
  \item Prove that for all composite $n$ apart from 6,
    \[\sqrt n\le\varphi(n)\le n-\sqrt n.\]
  \item Let $n$ be an even positive integer such that $\sigma(n)=2n$. Prove that
    $n=2^{p-1}\left(2^p-1\right)$, where $p$ is a prime.
  \item For any positive integer $n$, prove that $\displaystyle\sum_{d\mid
    n}\varphi(d)=n$.
\end{enumerate}
\section{Homework}
\begin{itemize}
  \item Complete the survey:
    \href{https://forms.gle/RYsmy3SBhQa9y4Yh9}{https://forms.gle/RYsmy3SBhQa9y4Yh9}
  \item Solve and submit any three problems from the previous section.
\end{itemize} 
\end{document}
