\documentclass{article}
\usepackage[inline]{asymptote}
\usepackage{enumitem,amsmath,amsfonts,geometry,parskip,amssymb,hyperref}
\hypersetup{colorlinks=true}
\def\asydir{asy}
\newcommand\lcm{\mathrm{lcm}}
\newcommand\Zz{\mathbb{Z}}
\newcommand\Nn{\mathbb{N}}
\newcommand\ord{\mathrm{ord}}
\title{Review and Extension: Constructions, Sequences, Polynomials}
\author{Andres Buritica Monroy}
\date{}
\begin{document}
\maketitle
\section{Key concepts for this term}
\begin{itemize}
  \item Existence results: infinitude of primes, Bezout, CRT, pigeonhole
  \item Techniques for controlling constructions
  \item Factor theorem, remainder theorem
  \item If $a$ and $b$ are integers, then $a-b\mid p(a)-p(b)$
  \item Finite differences
  \item Division algorithm for polynomials
  \item Rational root theorem
  \item Sequences and integer functions
\end{itemize}
\section{What Now?}
You now know all of the number theory that you need to solve problems at
the AMO and AMOC Senior Contest. To turn this knowledge into results, you will
need to familiarise yourself with this toolkit and how each of the tools within it
can be applied in different situations. For this, there is no substitute for
practice.

My Camp-Level course builds upon the foundation that we have laid in
Senior-Level. It includes a selection of more challenging problems to help you
practice using the results and techniques that we have covered this year, as
well as more advanced content that is necessary for solving harder
problems, such as those appearing on team selection tests and at the IMO\@.

Apart from the problems on these handouts, the best place to go for
practice problems is AoPS, especially
the pages for \href{https://artofproblemsolving.com/community/c58}{national
olympiads} and \href{https://artofproblemsolving.com/community/c59}{TSTs}.
However, note that solutions on AoPS are user-contributed and often incorrect.
\end{document}
