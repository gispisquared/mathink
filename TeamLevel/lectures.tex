\documentclass{article}
\usepackage[inline]{asymptote}
\usepackage{enumitem,amsmath,amsfonts,amssymb,geometry,parskip}
\def\asydir{asy}
\title{Team Level Lectures}
\newcommand\lcm{\mathrm{lcm}}
\newcommand\Zz{\mathbb{Z}}
\newcommand\ord{\mathrm{ord}}
\author{Andres Buritica Monroy}
\date{}
\begin{document}
\maketitle
\section{Farey sequences}
  Let $n$ be a fixed positive integer. Let
  $\frac{a_1}{b_1},\ldots,\frac{a_k}{b_k}$ be the rational numbers between 0
  and 1 inclusive with denominators at most $n$, written in increasing order
  and lowest terms.
  \begin{itemize}
    \item Prove that for each $i$, $a_{i+1}b_i-a_i b_{i+1}=1$.
    \item Prove that the rational number $x$ with smallest denominator such
      that $\frac{a_i}{b_i}<x<\frac{a_{i+1}}{b_{i+1}}$ is
      $\frac{a_i+a_{i+1}}{b_i+b_{i+1}}$. 
    \item Which pairs of numbers appear as consecutive $b_i$s?
  \end{itemize}

  Example problems:
  \begin{itemize}
    \item 
      Suppose that $(a_1,b_1),(a_2,b_2),\ldots,(a_{100},b_{100})$ are distinct
        ordered pairs of nonnegative integers.
        Let $N$ denote the number of pairs of integers $(i,j)$ satisfying
        $1\leq i<j\leq 100$ and $|a_i b_j-a_j b_i|=1$. 
        Determine the largest possible value of $N$ over all possible choices of the
        100 ordered pairs.
    \item A lattice point in the Cartesian plane is a point whose coordinates
      are both integers. A lattice polygon is a polygon all of whose vertices
      are lattice points.

      Let $\Gamma$ be a convex lattice polygon. Prove that $\Gamma$ is contained
      in a convex lattice polygon $\Omega$ such that the vertices of $\Gamma$
      all lie on the boundary of $\Omega$, and exactly one vertex of $\Omega$ is
      not a vertex of $\Gamma$.
  \end{itemize}
\section{Dirichlet Convolution and Mobius Inversion}
  Let $f:\mathbb N\to\mathbb R$ and $g:\mathbb N\to\mathbb R$ be two functions.
  We define the \emph{Dirichlet convolution} $f*g$ as
  \[(f*g)(n)=\sum_{d\mid n}f(d)g\left(\frac nd\right).\]

  We define the functions $d,\ \sigma,\ \varphi$ as before and also define the
  functions \[\zeta(n)=1,\ \psi(n)=n.\]
  \begin{itemize}
    \item Prove that $*$ is associative: that is,
      $(a*b)*c=a*(b*c)$.
    \item Prove that if $a$ and $b$ are multiplicative then so is $a*b$.
    \item Find a function $\delta$ such that $\delta*a=a$ for all functions $a$.
    \item Find a function $\mu$ such that $\mu*\zeta=\delta$.
    \item Prove that $g=f*\zeta\iff f=g*\mu$.
    \item Find $\zeta*\zeta,\ \psi*\zeta$ and $\varphi*\zeta$.
    \item Prove that
      \[\sum_{i=1}^n f(i)\left\lfloor\frac
          ni\right\rfloor=\sum_{j=1}^n(f*\zeta)(j).\]
  \end{itemize}

  Example problems:

  For a positive integer $n$, let $f(n)$ be the number of binary strings of length $n$ that
      can't be expressed as an $m$-fold repetition of another binary string for
      any $m>1$.

      For example, $f(6)=54$ since the only strings of length 6 that can be
      expressed as an $m$-fold repetition of another binary string for some $m>1$
      are 000000, 001001, 010010, 010101, 011011, 100100, 101010, 101101, 110110,
      111111.
       
      \begin{itemize}
        \item Find two functions $g$ and $h$, in closed form, such that $f=g*h$.
        \item 
          Prove that $n\mid f(n)$.
        \item Find all $n$ for which $n\mid\displaystyle\sum_{i=1}^n f(i)\left\lfloor\frac
              ni\right\rfloor$.
      \end{itemize}
\section{Polynomials mod $p$}
Let $p$ be prime.
\begin{itemize}
  \item Prove that unique factorisation holds for polynomials mod $p$. (This is
    not true for all integers --- for instance,
    $(x-1)^2\equiv(x-3)^2\pmod 4$.)
  \item Prove that for every function $f:\Zz_p\to\Zz_p$ there is a unique polynomial $P$ in
    $\Zz_p$ of degree less than $p-1$ such that $f(x)=P(x)$ for each
    $x\in\Zz_p$.
  \item Let $g$ be a generator mod $p$, and let $ab=p-1$. Prove that
    \[\prod_{i=1}^a (x-g^{bi})\equiv x^a-1\pmod p.\]
    What does this tell us about the roots of the cyclotomic polynomials in mod
    $p$?
  \item Consider all $\binom{p-1}k$ products of $k$ elements of $\Zz_p$. Prove
    that their sum is divisible by $p$.
  \item For any positive integer $n<p-1$, prove that
    \[\sum_{i=1}^{p-1} i^n\equiv 0\pmod p.\]
\end{itemize}
Example problems:
\begin{itemize}
  \item Let $p$ be an odd prime. We compute the product of $(4-x)$,
    where $x$ varies over all residues mod $p$ except the quadratic residues.
    Find the least residue of this product mod $p$.
  \item Find the least residue of the sum of all generators mod $p$.
  \item Let $\mathbb{Z}/n\mathbb{Z}$ denote the set of integers considered
    modulo $n$ (hence $\mathbb{Z}/n\mathbb{Z}$ has $n$ elements). Find all
    positive integers $n$ for which there exists a bijective function $g:
    \mathbb{Z}/n\mathbb{Z} \to \mathbb{Z}/n\mathbb{Z}$, such that the 101
    functions
    \[g(x), \quad g(x) + x, \quad g(x) + 2x, \quad \dots, \quad g(x) + 100x\]are
    all bijections on $\mathbb{Z}/n\mathbb{Z}$.
  \item Let $p$ be an odd prime. An integer $x$ is called a quadratic
    non-residue if $p$ does not divide $x-t^2$ for any integer $t$.

    Denote by $A$ the set of all integers $a$ such that $1\le a<p$, and both $a$
    and $4-a$ are quadratic non-residues. Calculate the remainder when the
    product of the elements of $A$ is divided by $p$.
\end{itemize}
\section{Binomial coefficients mod $p$}
\begin{itemize}
  \item Wolstenholme's Theorem:
    let $a$ and $b$ be positive integers, and let $p$ be a prime greater
    than 3. Prove that \[\binom{ap}{bp}\equiv\binom ab\pmod {p^3}.\]
  \item Lucas' Theorem: let $m=\sum m_i p^i$ and $n=\sum n_i p^i$ be the base-$p$ expansions of
    $m$ and $n$, where $p$ is prime. Prove that
    \[\binom mn\equiv\prod{\binom{m_i}{n_i}}\pmod p.\]
\end{itemize}
\section{Weak Prime Number Theorem}
\begin{enumerate}
  \item Prove that the sum of the reciprocals of the primes diverges.
  \item
    Let $n$ be a positive integer larger than 1.
    \begin{enumerate}
      \item Prove that the product of all primes between $\left\lceil\frac
        n2\right\rceil$ and $n$
        (including $n$, not including $\left\lceil\frac n2\right\rceil$)
        is less than
        $2^n$.
      \item Prove that the product of all primes between 1 and $n$ is at most
        $4^{n-1}$.
      \item Find some real number $c$ independent of $n$ such that there are at
        most $\frac{cn}{\log_2 n}$ primes that are at most $n$.
    \end{enumerate}
  \item 
    Let $n$ be a positive integer larger than $2^{2^{2^2}}$.
    \begin{enumerate}
      \item Let $p$ be a prime.
        \begin{itemize}
          \item Prove that if $p^k\mid\binom{2n}n$ then $p^k<
            2n$.
          \item Prove that if $2p\le 2n< 3p$ then $p\nmid\binom{2n}n$.
        \end{itemize}
      \item Prove that 
        \[\prod_{\substack{p^k\|\binom{2n}n\\ p\le n}}p^k<\binom{2n}n.\]
      \item Find some real number $c$ independent of $n$ such that there are at
        least $\frac{cn}{\log_2 n}$ primes that are at most $n$.
    \end{enumerate} 
\end{enumerate}
\end{document}
